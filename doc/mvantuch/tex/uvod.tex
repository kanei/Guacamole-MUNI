\chapter{Úvod}
Tato práce se zabývá projektem, jehož cílem je vytvoření informačního systému pro studenty Ekonomicko-správní fakulty Masarykovy Univerzity. Stávající řešení v podobě webové stránka bylo postaveno na \gls{opensource} projektu \emph{Drupal}\footnote{https://drupal.org} verze 6 a celá jeho stuktura byla řešena na bázi slovníků a stránek, bez rozlišení typů obsahu. Mgr. Ondřej Materna ve své práci zanalyzoval možnosti zlepšení řešení a reálné požadavky studentů. Výsledkem je návrh řešení, který se neslučoval s existujícími stránkami, které tak musely být razantně přepracovány. \emph{Drupal} verze 6 existoval již více než pět let a dva roky existuje i jeho aktualizovaná verze 7\cite{website:wiki:drupal}. Ten poskytuje vyšší rychlost, stabilitu i rozšíření díky širší podpoře komunity a nezávislých vývojářů. Stávající řešení je hlouběji rozebráno v kapitole~\ref{chap:analyza}, ve které jsou zároveň popsány technologie využité k implementaci nové verze portálu a základní architektura řešení.

Důležitou částí portálu je propojení s \gls{aspi} za pomoci klienta vzdálené plochy. Ve stávajícím řešení byli studenti nuceni používat jednu ze dvou možností připojení:

\begin{enumerate}
  \item lokální instalace nativní aplikace \gls{aspi} na klientský počítač a  její spuštění - odkazy se otevírají přímo v aplikaci
  \item připojení se na vzdálenou plochu pomocí jednoho z veřejně dostupných klientů a používání nativní aplikace zde
\end{enumerate}

Odkazy na stránkách se však automaticky nepřenášely na vzdálenou plochu a celkově vyžadovalo toto řešení vyšší technickou zdatnost uživatelů.

Hlavním cílem projektu je uživatelská přívětivost a proto byl při výběru řešení kladen důraz převážně na jednoduchost a minimální požadavky na klientské zařízení a uživatele. Bylo zvoleno řešení postavené na prvcích jazyka HTML5 a open-source nástroji \emph{Guacamole}. Ten poskytuje klienta vzdálené plochy čistě skrze okno prohlížeče. Komunikace se vzdálenou plochou probíhá za pomoci sprostředkovatelského proxy serveru. Architektura je detailně popsána v kapitole \ref{chap:implementace-guacamole}.

Jak bylo zmíněno výše, tato práce staví na diplomové práci Mgr. Ondřeje Materny ,,Návrh a realizace právního portálu pro ESF MU''\cite{omaterna2013}. Její obsah je důkladně zanalyzován a z ní vyplývající poznatky uplatněny na prostředí \gls{cms} \emph{Drupal} a jeho možnosti. Proces aktualizace a implementace funkční a základ vzhledové stránky řešení jsou popsány v kapitole  \ref{chap:implementace-drupal}. Vzhled samotný není předmětem této práce, nýbrž práce Bc. Ivany Haraslínové. 

Z důvodu spolupráce mezi více studenty byl vytvořen základní systém pro vedení projektu, využívající prostředí systému \emph{GitHub} a jeho stručný popis je obsažen v kapitole~\ref{chap:vyvoj}. Projekt je veřejně dostupný na adrese \url{https://github.com/kanei/esf-mu-portal}.
