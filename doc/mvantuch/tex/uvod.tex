\chapter{Úvod}
Cílem celého projektu je vytvoření informačního systému pro studenty Ekonomicko-správní fakulty Masarykovy Univerzity. Před započatím tého práce existovala webová stránka postavená na Drupalu verze 6, ve které byla vytvořena komplexní struktura obsahu - právních oborů a zákonů. Kolega Ondřej Materna ve své práci zanalyzoval toto řešení a reálné požadavky studentů. Výsledkem jeho práce bylo řešení, které se příliš neslučovalo s aktuálně existujícími stránkami. Drupal verze 6 již existoval již více než pět let a dva roky existoval i jeho nástupce, drupal verze 7\cite{website:wiki:drupal}. Ten poskytoval mnohem rozsáhlejší možnosti rozšíření, neboť se prakticky všechen vývoj přesunul k němu. 

Důležitou částí portálu je propojení s \gls{aspi} za pomocí vzdálené plochy. V době tvorby této práce byli všichni studenti nuceni používat jednu ze dvou možností, jak se připojit na portál \gls{aspi}. Prvním z nich byla lokální instalace na počítač a jeho přímé spuštění, zatímco druhou připojení na vzdálenou plochu a používání portálu zde. Odkazy na stránkách se však automaticky nepřenášely na vzdálenou plochu a celkově vyžadovalo toto řešení vyšší technickou zdatnost uživatelů.

Protože hlavním cílem projektu je uživatelská přívětivost, byl při výběru řešení kladen důraz převážně na jednoduchost a minimální požadavky na klienta - ať již uživatele, nebo jeho stroj. Bylo zvoleno řešení postavené na moderním HTML5 a open-source řešení Guacamole. To poskytuje klienta zobrazeného čistě v okně prohlížeče, komunikujícího se vzdálenou plochou za pomocí proxy serveru přeposílajícího požadavky dále. Architektura je detailně popsána v kapitole \ref{chap:implementace}.

Tato práce staví na práci Mgr. Ondřeje Materny ,,Návrh a realizace právního portálu pro ESF MU'' \cite{omaterna2013}. Ta je důkladně zanalyzována a její závěry uplatněny na prostředí CMS Drupal a jeho možnosti. Je popsáno rozdělení na jednotlivé sektory stránek, vzhled samotný však není předmětem této práce, nýbrž %% TODO doplnit Skarlet
Z důvodu spolupráce mezi více studenty byl vytvořen základní systém pro 

%TODO Popsat co studenti museji aktualne delat k tomu aby se dostali k ASPI a jak to zmenime
