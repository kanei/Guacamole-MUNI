\chapter{Závěr}

Projekt implementace informačního portálu pro fakultu \gls{esf} byl vyvíjen za využití moderních technoglogií. Ačkoliv je většina nastavení \gls{cms} \emph{Drupal} ukládaná v databázi a v typických případech je jediným způsobem replikace prostředí jeho zkopírování, byly využity moduly \emph{Features} a \emph{Profile Builder} k vytvoření udržitelného procesu vývoje, který lze využít i do budoucna. Vývoj probíhal na odděleném prostředí (lokálně běžící server) a veškeré změny byly prováděny buď v modulu specifickém pro tento projekt (\emph{ESF Module}), nebo byly exportovány za pomoci \emph{Features} (\emph{ESF Feature}). Vývoj probíhal za maximálního respektování standartů, včetně stylu kódu a komentářů. 

Všechny využité moduly třetích stran byly průběžně zaznamenávány v instalačním profilu (\emph{ESF Profile}) a díky této kombinaci je možné nainstalovat v případě nutnosti celý projekt zcela od nuly bez ovlivnění historickými chybami, které mohou vznikat například špatnými daty v databázi. Pro zjednodušení procesu aktualizace a instalace byly sepsány instalační skripty za pomocí nástrojů \emph{Phing} a \emph{Maven}, které byly zároveň propojeny s \gls{ci} 
serverem \emph{Team City}, umožňujícím průběžnou kontrolu kvality a funkčnosti kódu.

Portál byl vyvíjen s ohledem na moderní trendy v interaktivních webových aplikací, včetně optimalizace pro dotyková zařízení a přístroje s nižšími zobrazovacími schopnostmi. Tyto požadavky byly zohledněny i při implementaci administrace, která byla zároveň vytvořena s co největším ohledem na uživatelskou přívětivost a přímočarost. Do systému je možné vkládat většinu typů multimediálního obsahu s možností rozšíření do budoucna dle možných měnících se požadavků uživatelů systému (je možné například připojovat videa z \emph{YouTube} a podobně).

Struktura portálu byla vyvinuta dle požadavků vycházejících z práce Mgr. Ondřeje Materny, které byly průběžně konzultovány s JUDr. Jindřiškou Šedovou, CSc. a upravovány dle potřeby. Výsledkem je návrh reflektujícíc strukturu výuky a poskytující co možná nejkonzistentěnjší pohled na informace k předmětům a oborům práva, které tvoří základní stavební kameny systému. Na úvodní straně byl vytvořen diagram, který dále zviditelňuje propojení mezi předměty a obory práva, čímž uživatelům poskytuje jednoduchou možnost procházet pouze relevantní informace. Propojení jsou zobrazena i na stránkách obou výše zmíněných entit a umožňují rychlou navigaci mezi souvisejícími tématy. 

Do portálu bylo implementováno propojení s aplikací \emph{Guacamole} a jejím systémovým démonem \emph{Guacd}, které dohromady umožňují připojení se ke vzdálené ploše skrze internetový prohlížeč podporující technologie \emph{HTML5} a \emph{CSS3}. Díky této možnosti se prudce zjednodušuje interakce uživatelů s aplikací. Uživatelé se přihlašují za pomoci údajů synchronizovaných s \gls{is} \gls{muni} a jsou upozorněni na nutnost vyplnění přístupových údajů, potřebných k připojení ke vzdálené ploše. Případně jim je nabídnuta možnost přesměrování na stránku, která jim umožní vygenerování hesla nového. Po přihlášení již nejsou žádána žádná další hesla. Tím se uživatelská zkušenost značně vylepšuje a odpadá nutnost instalace klientů pro připojení ke vzdálené ploše a opakujícího se zadávání hesla, která mohla být jinak pro velkou skupinu méně technicky zdatných uživatelů značně frustrující. Pro uživatele preferující připojení skrze dedikovaného klienta je tato možnost zachována.

V době vypracování této práce již portál běžel v testovacím provozu a byly opravovány poslední chyby před finálním spuštěním a zpřístupněním studentům \gls{esf} Masarykovy univerzity. Řešení propojení mezi \gls{cms} \emph{Drupal} a nástrojem \emph{Guacamole} je dosud nevyužívanou cestou a díky skutečnosti, že je projekt veřejně dostupný na portálu \emph{GitHub}, byl již zdrojový kód použit i mimo toto řešení.
