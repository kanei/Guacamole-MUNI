\chapter{Instalace, konfigurace a správa portálu}
\label{chap:instalace}

Díky využití Features~\ref{nic} a instalačních profilů je možné portál nainstalovat na jakýkoliv server bez kopírování souborů, či databáze, pouze za pomocí nástrojů Phing a Drush. Je vytvořena kompletní struktura obsahu, nastaven odpovídající vzhled a vytvořeny uživatelské role. Obsah však přenášen není a je nutné jej buď importovat, nebo vytvořit znovu. Tento přístup se nejvíce hodí v případě vývoje nových prvků, či správy funkcionality vzdálené plochy.

\section{Požadavky a příprava systému}

Na stroji je nutné mít nainstalovaný internetový server (Apache, Lighttpd, atd.) s podporou PHP. MySQL potřebné není, lze využít minimalistickou implementaci SQLite, která data ukládá do jediného souboru ve složce \texttt{sites/default/files}. Drupal dále ke svému chodu vyžaduje knihovny php php-gd a php-pdo. Pro usnadnění instalace je nutné mít nainstalovánu knihovnu \ref{phing}.

Volitelně je možné mít nainstalováno: ??

Obsah repozitáře projektu je vhodné umístit například do domovské složky - typicky příkladem \texttt{git clone git@github.com:kanei/esf-mu-portal.git}. Před samotnou instalací je nutné nastavit základní vlastnosti projektu. Soubor \texttt{phing/default.build.properties} překopírujte do hlavní složky repozitáře (např.\texttt{\~/esf-mu-portal}) a přejmenovat jej na build.properties, nebo spustit příkaz \texttt{\$~phing}, který provede úvodní inicializaci, jejíž součástí je i překopírování souboru v případě, že neexistuje. V konfiguračním souboru je nutné v proměnné \texttt{project.dir} nastavit cílovou složku pro instalaci. Složka však nesmí existovat a uživatel spouštějící skript musí mít práva k jejímu vytvoření, protože jinak Drush není schopen s instalací pokračovat a skončí s chybovou hláškou. V konfiguračním souboru lze změnit další nastavení, jakými jsou administrátorský účet, url databáze a pod, ale pro základní funkcionalitu je možné tyto hodnoty ponechat nezměněné.

\section{Postup instalace}
Instalace probíhá za využití dávkového instalačního souboru \texttt{build.xml}, ve kterém je krom jiného popsán instalační cíl \emph{install}. Pro spuštění instalace je potřeba v terminálu spustit příkaz \texttt{\$~phing install} a v  průběhu instalace se provedou následující kroky:

\begin{enumerate}
  \item Je zkontrolována cílová složka a uživatel je případně upozorněn na nutnost jejího smazání.
  \item \emph{drush make} - Vytvoří se stuktura stránek, při čemž se ze souboru \texttt{esf.make} (který načte podsoubor \texttt{src/profiles/esf\_profile/esf\_profile.make}) načte seznam všech modulů využívaných v řešení spolu s jejich požadovanou verzí. Tyto se stáhnou do cílové složky, ve které se zároveň připraví konfigurační soubory pro instalaci Drupalu.
  \item Do projektového adresáře jsou překopírovány moduly, témata a instalační profily projektu, aby s nimi mohl instalační skript pracovat.
  \item Při instalaci nejsou automaticky provedeny všechny úkony a některá nastavení je nutné spustit manuálně po dokončení. Jedná se mimo jiné o nastavení tématu vzhledu.
  \item \emph{drush site-install} - Spustí se instalace Drupalu z příkazové řádky, při které se povolí základní moduly a vytvoří databázová struktura. Je nainstalován instalační profil \emph{esf\_profile} a s ním jsou povoleny moduly rozšiřující funkcionalitu a také rysy dokončující nstavení stránek. Celý proces trvá až několik minut a v průběhu nijak uživatele neinformuje o průběhu, je tedy důležité vyčkat až do jeho ukončení.
\end{enumerate}

Po instalaci je vhodné zkontrolovat práva souborů a nastavit uživatele i skupinu na hodnoty daného internetového serveru (apache/lighttpd/...) za pomocí příkazů \texttt{\$~sudo~chown~apache~esf~-R} a \texttt{\$~sudo~chgrp~apache~esf~-R}. Toto je obzvláště důležité v případě použití SQLite databáze, ke které jinak nemá drupal práva zapisovat a tím pádem stránky spadnou s fatální chybou. 

Pokud vše proběhlo v pořádku a server je schopen přistupovat ke všem souborům i databázi, je možné na adrese propojené se stránkami navštívit úvodní stránku - typicky se jedná o \emph{http:\\localhost/esf}. Na stránky bylo převedeno veškeré nastavení, ale nebyl zde vytvořen žádný obsah. Ten lze buď migrovat z produkčních stránek, nebo generovat za pomocí modulu Devel. Na adrese \emph{http://localhost/esf/?q=user} se lze přihlásit za pomocí administrátorského účtu (pokud nebyl změněn v konfiguračním souboru, jedná se o admin/admin) a přistoupit k administračnímu rozhraní stránek.

\section{Administrace}
Správa probíhá za využití klasické administrace v jádře Drupalu, rozšířené Pracovní stůl a další pohledy vytvořené specificky pro tento projekt. 
