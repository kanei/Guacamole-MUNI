\chapter{Instalace, konfigurace a správa portálu}
\label{chap:instalace}

Díky využití Features (sekce \ref{subsec:features}) a instalačních profilů (sekce \ref{subsec:drupal}) je možné portál nainstalovat na jakýkoliv server bez kopírování souborů či databáze, pouze za pomocí nástrojů \emph{Phing} a \emph{Drush}. Je vytvořena kompletní struktura obsahu, nastaven odpovídající vzhled a vytvořeny uživatelské role. Obsah však přenášen není a je nutné jej buď importovat, nebo vytvořit znovu. Tento přístup se nejvíce hodí v případě vývoje nových prvků, či správy funkcionality připojení ke vzdálené ploše.

\section{Požadavky a příprava systému}
\label{sec:priprava-systemu}

Na stroji je nutné mít nainstalovaný internetový server (\emph{Apache}, \emph{Lighttpd}, etc.)\footnote{Technologie jsou popsány v sekci~\ref{sec:technologies}.} s podporou \emph{PHP} a \emph{JAVA} \gls{servlet} kontejner \emph{Apache Tomcat}. Systém podporuje \emph{MySQL} a \emph{PostgreSQL}, pro ulehčení instalace lze však využít i minimalisticky orientovanou implementaci \emph{SQLite}, která data ukládá do jediného souboru ve složce \texttt{sites/default/files}. \emph{Drupal} dále ke svému chodu vyžaduje knihovny php \emph{php-gd}, {php-pdo} a {php-krb5\footnote{https://github.com/ashishtilara/php\_krb5}}. Pro usnadnění instalace je nutné mít nainstalovánu knihovnu \emph{Phing}. Pro zkompilování systémového démona je potřeba mít nainstalovány nástroje \emph{gcc}, \emph{make}, \emph{Maven} a \emph{JRE} a knihovny \emph{libpng-devel}, \emph{cairo-devel} a \emph{freerdp-devel}. 

Volitelně je možné mít nainstalován \emph{Drush} a s ním spojená nastavení prostředí systému, určená k urychlení manuálních operací. Ještě donedávna se \emph{Drush} instaloval jako knihovna pro \emph{PEAR}, tento způsob však byl nahrazen instalací pomocí programu \emph{Composer}, který je tím pádem potřeba před instalací. Nejjednodušší způsob, jak jej nainstalovat, je pomocí \emph{CURL}:

\begin{lstlisting}[language=bash]
  $ curl -sS https://getcomposer.org/installer | php
\end{lstlisting}

Jakmile je composer dostupný, lze již Drush nainstalovat v jeho 6. verzi:

\begin{lstlisting}[language=bash]
  $ composer global require drush/drush:6.*
\end{lstlisting}

Po instalaci je vhodné nastavit prostředí systému a integraci mezi nástroji \emph{Drush} a \emph{BASH}. Do souboru \texttt{.bashrc} v domovském adresáři vložte řádek obsahující cestu k Drushi a kód s nahráním souboru .drush\_bashrc.

\begin{lstlisting}[language=bash]
export PATH="$HOME/.composer/vendor/bin:$PATH"

if [ -f ~/.drush_bashrc ] ; then
  . ~/.drush_bashrc
fi
\end{lstlisting} 

Je vhodné aktualizovat bashrc soubor:

\begin{lstlisting}[language=bash]
  $ source ~/.composer/vendor/drush/drush/examples/
    example.bashrc
\end{lstlisting} 

Nyní by měly být přístupné příkazy pod zkrácenými aliasy - například \texttt{dr} místo \texttt{drush} a \texttt{cc} namísto \texttt{drush cache-clear}. Je možné i povolit automatické doplňování příkazů, dle instrukcí na stránkách vývojového týmu. Dále je vhodné v souboru \linebreak\texttt{\$HOME/.drush/aliases.drushrc.php} definovat alias stránky pro usnadnění manipulace. Obsah souboru může vypadat následovně:

\begin{lstlisting}[language=php]
  <?php
  $aliases['pravo'] = array(
    'root' => '/srv/www/htdocs/',
    'uri' => 'pravo.com',
  );
\end{lstlisting}

Po tomto nastavení je možné do složky přistoupit jednoduše pomocí příkazu \texttt{\$~cddl~@pravo} (případně \texttt{\$~cd~@pravo} pokud je nastavený alias pro příkaz \texttt{cd}). Všechny příkazy nástroje \emph{Drush} je možné spouštět pro daný alias, což je výhodné v případě více prostředí (vývojové vs. produkční).

Obsah repozitáře projektu může být umístěn kdekoliv, například v domovské složce uživatele:

\begin{lstlisting}[language=bash]
  $ cd ~
  $ git clone git@github.com:kanei/esf-mu-portal.git  
\end{lstlisting}

Před samotnou instalací je nutné nastavit základní vlastnosti projektu. Soubor \texttt{phing/default.build.properties} překopírujte do hlavní složky repozitáře a přejmenujte jej na \texttt{build.properties}:

\begin{lstlisting}[language=bash]
  $ cp ~/esf-mu-portal/phing/default.build.properties 
    ~/esf-mu-portal/build.properties
\end{lstlisting}

Druhou možností je spuštění úvodní inicializace, jejíž součástí je i překopírování souboru v případě, že neexistuje:

\begin{lstlisting}[language=bash]
  $ phing void
\end{lstlisting}

V konfiguračním souboru je nutné v proměnné \texttt{project.dir} nastavit cílovou složku pro instalaci. Složka však nesmí existovat a uživatel spouštějící skript musí mít práva k jejímu vytvoření, protože jinak \emph{Drush} není schopen s instalací pokračovat a skončí s chybovou hláškou. V konfiguračním souboru lze změnit další nastavení, jakými jsou administrátorský účet, url databáze a pod, ale pro základní funkcionalitu je možné tyto hodnoty ponechat nezměněné.


\section{Nasazení (deployment) a jeho možnosti}

Pro správu \gls{deployment} je využíván nástroj \emph{Phing}, umožňující definování pravidel v souboru \texttt{build.xml}. Kromě tohoto souboru je také dostupný soubor \linebreak \texttt{phing/\-default.build.properties} obsahující výchozí nastavení projektu. Při první inicializaci jakéhokoliv cíle (\texttt{<target>}) se zkontroluje přítomnost souboru \linebreak \texttt{build.properties} a v případě jeho absence se na jeho místo zkopíruje výše zmíněný výchozí soubor. Je také možné spustit příkaz \texttt{phing void} který pouze připraví tento konfigurační soubor. 

V konfiguračním souboru je vhodné zadat nastavení prostředí, na kterém se bude portál instalovat (cesty k souborům, nastavení databáze a pod.) - více informací je v samotném souboru.

\begin{description}
  \item[void] \hfill \\
  Pouze provede inicializaci konfiguračních souborů a ukončí se.  
  \item[guacamole] 
  Vytvoří konfigurační soubory projektu Guacamole ve správných složkách (je zkopírován soubor \texttt{phing/default.guacamole.properties}.
  \item[deploy]
  Překopíruje aktuální moduly na správné místo a zajistí správné nastavení portálu dle aktuálního nastavení (převážně využití Features).
  \item[install]
  Provede kompletní instalaci portálu dle nastavení v konfiguračních souborech - překopíruje moduly a témata specifická pro toto řešení, stáhne potřebné zdrojové kódy Drupalu z internetu a nainstaluje Drupal, kterému následně nastaví správné téma vzhledu a povolí potřebné moduly.
\end{description}

\section{Instalace informačního systému (Drupal)}
Instalace probíhá za využití dávkového instalačního souboru \texttt{build.xml}, ve kterém je krom jiného popsán instalační cíl \emph{install} (viz předchozí kapitola). Pro spuštění instalace je potřeba v terminálu spustit následující příkaz:

\begin{lstlisting}[language=bash]
  $ phing install
\end{lstlisting}

\subsection*{Průběh instalace}

\begin{enumerate}
  \item Je zkontrolována cílová složka a uživatel je případně upozorněn na nutnost jejího smazání.
  \item \lstinline[language=bash]{$ drush make} - Vytvoří se stuktura stránek, při čemž se ze souboru \texttt{esf.make} (který načte podsoubor \texttt{src/profiles/esf\_profile/esf\_profile.make}) načte seznam všech modulů využívaných v řešení spolu s jejich požadovanou verzí. Tyto se stáhnou do cílové složky, ve které se zároveň připraví konfigurační soubory pro instalaci Drupalu.
  \item Do projektového adresáře jsou překopírovány moduly, témata a instalační profily projektu, aby s nimi mohl instalační skript pracovat.
  \item Při instalaci nejsou automaticky provedeny všechny úkony a některá nastavení je nutné spustit manuálně po dokončení. Jedná se mimo jiné o nastavení tématu vzhledu.
  \item \lstinline[language=bash]{$ drush site-install} - Spustí se instalace Drupalu z příkazové řádky, při které se povolí základní moduly a vytvoří databázová struktura. Je nainstalován instalační profil \emph{esf\_profile} a s ním jsou povoleny moduly rozšiřující funkcionalitu a také rysy dokončující nstavení stránek. Celý proces trvá až několik minut a v průběhu nijak uživatele neinformuje o průběhu, je tedy důležité vyčkat až do jeho ukončení.
\end{enumerate}

Po instalaci je vhodné zkontrolovat práva souborů a nastavit uživatele i skupinu na hodnoty daného internetového serveru (apache/lighttpd/\dots)

\begin{lstlisting}[language=bash]
  $ sudo chown apache esf -R 
  $ sudo chgrp apache esf -R
\end{lstlisting}

Toto je obzvláště důležité v případě použití \emph{SQLite} databáze, ke které jinak nemá drupal práva zapisovat a tím pádem stránky spadnou s fatální chybou. 

Pokud vše proběhlo v pořádku a server je schopen přistupovat ke všem souborům i databázi, je možné na adrese propojené se stránkami navštívit úvodní stránku (např. \emph{http:/localhost/esf}). Na stránky bylo převedeno veškeré nastavení, ale nebyl zde vytvořen žádný obsah. Ten lze buď migrovat z produkčních stránek, nebo generovat za pomocí modulu \emph{Devel}. Na adrese \emph{http://localhost/esf/?q=user} se lze přihlásit za pomocí administrátorského účtu (pokud nebyl změněn v konfiguračním souboru, jedná se o účet admin a heslo admin) a přistoupit k administračnímu rozhraní stránek.

\section{Instalace Guacamole}
Guacamole vyžaduje na straně serveru dvě komponenty: \emph{JAVA} \gls{servlet} aplikaci a démona \emph{Guacd}. Druhý jmenovaný není nijak upraven a lze tedy stáhnout aktuální verzi ze stránek projektu\footnote{V době sepsání tohoto textu byla aktuální verze 0.9, na které byl i celý projekt testován. Projekt je dostupný na adrese http://guac-dev.org/release/release-notes-0-9-0}. Po stažení stačí zkompilovat a nainstalovat projekt (za předpokladu, že jsou v systémy dostupný všechny knihovny a nástroje zmíněné v kapitole~\ref{sec:priprava-systemu}):

\begin{lstlisting}[language=bash]
  $ cd guacamole-server-0.9.0
  $ ./configure
  $ make
  $ make install
  $ ldconfig
\end{lstlisting}

V této chvíli by již měl jít spustit příkaz \texttt{guacd}, který spustí na portu 4822 službu pro připojení ke vzdálené ploše. Nyní je potřeba nastavit parametry připojení. Na stránce portálu, na adrese \texttt{admin/config/esf/settings} je nutno nastavit správné hodnoty a zajistit přístup ke konfiguračnímu souboru Guacamole pro \emph{Drupal} i \emph{Guacamole} \emph{JAVA} aplikaci. Dále budou předpokládány výchozí hodnoty nastavení.

\begin{lstlisting}[language=bash]
  $ cp ~/esf-mu-portal/phing/default.guacamole.properties /srv/guacamole/guacamole.properties
  $ chown apache guacamole.properties 
  $ chgrp apache guacamole.properties 
\end{lstlisting}

Jméno a uživatelská skupina závisí na použitém operačním systému. V souboru guacamole.properties je nutno nastavit parametry připojení ke vzdálené ploše \gls{aspi} a informace o démonu Guacd, což lze udělat buď přímo v souboru, nebo za pomocí Drupalu na adrese \texttt{admin/config/esf/remote}. 

Tím je dokončeno nastavení démona a lze přikročit k instalaci \emph{JAVA} \gls{servlet}. Ten je potřeba zkompilovat a nahrát jako do \emph{Apache Tomcat}.

\begin{lstlisting}[language=bash]
  $ cd ~/esf-mu-portal/
  $ mvn package
  $ cp ~/esf-mu-portal/target/guacamole-esf-0.4.war /srv/tomcat/webapps/
  $ rctomcat restart

\end{lstlisting}

Po dokončení kompilace by měla být aplikace automaticky dostupná na adrese \texttt{localhost:8080/guacamole-esf-[číslo verze]/}. Tuto adresu je nutné zadat do políčka \emph{Guacamole Servlet path} v portálu na adrese \\ \texttt{admin/config/esf/settings}. Poté by se již uživatelé měli být schopni přihlásit ke vzdálené ploše na adrese \texttt{/aspi}.

\section{Administrace}
\label{sec:administrace}
Správa probíhá za využití administrace dodávané v jádře Drupalu, rozšířené modulem Workbench a dalšími úpravami. Pro správu portálu jsou definovány čtyři úrovně uživatelských práv:

\begin{itemize}
  \item \textbf{anonymní uživatel} \hfill \\
    Tato role je spolu s rolí \emph{přihlášeného uživatele} definována v jádře \gls{cms} a nelze tedy měnit. Označuje uživatele, kteří se nepřihlásili do portálu. S ohledem na nutnost autentikace jsou tito uživatelé odstřiženi od přístupu k jakýmkoliv datům a je jim zpřístupněna pouze stránka s přihlášením.
  \item \textbf{přihlášený uživatel} \hfill \\
    Po přihlášení uživatelé získávají přístup k veškerému hlavnímu obsahu stránek, uživatelskému nastavení a připojení ke vzdálené ploše. 
  \item \textbf{moderátor} \hfill \\
    Moderátor má přístup k základní administraci stránek, obzvláště k úpravám obsahu, základním změnám provázání obsahu a administraci přístupu uživatelů.
  \item \textbf{administrátor} \hfill \\
    Pro tohoto uživatele neexistuje speciální role, ale jedná se o uživatele s identifikačním číslem 1, který má právo přístupu ke všem prvkům obsahu a administrace portálu a tato práva nejdou nijak omezit.
\end{itemize}

Tato práce se zabývá pouze specifickou administrací a nezaobírá se administrací Drupalu, ve které nebyly provedeny změny výraznějšího charakteru a funguje dle standartního přístupu. Po přihlášení má uživatel s rolí moderátora dostupný administrační panel umistitelný buď v horní, nebo levé části okna prohlížeče (obrázek~\ref{fig:workbench}). Administrační panel obsahuje odkazy na administrační stránky a pracovní stůl. 

\subsection*{Struktura pracovního stolu}
\begin{description}
  \item[Můj obsah] \hfill \\
  Tato sekce nabízí pohled na stránky vytvořené či upravené právě přihlášeným uživatelem a na nejaktuálnější obsah portálu všeobecně. Všechny pohledy vedou po rozkliknutí na detailnější přehledy obsahu dané sekce.  
  \item [Vytvořit obsah] \hfill \\
  V této sekci je možné vytvářet veškerý obsah, ke kterému má uživatel přístup - buď mediální obsah (obrázky, videa, dokumenty), ke kterému lze později přistupovat skrze \gls{wysiwyg} editor, nebo vlastní obsah portálu - nové předměty, stránky, obory práva viz. sekce~\ref{subsec:typy-obsahu}.
  \item [Seznam souborů] \hfill \\
  Poskytuje náhled na všechny soubory nahrané na portál, které jsou sledovány systémem a dostupné skrze administrační rozhraní.
  \item [Obory práva] \hfill \\
  Pohled na všechny zadané obory práva, jejich základní atributy a předměty, které se k nim vztahují. S obory práva lze manipulovat pomocí rozhraní využívajícího modulu Views Bulk Operations (sekce~\ref{subsec:vbo}) a tím pádem hromadnou úpravu entit.
  \item [Předměty] \hfill \\
  Podobně jako předchozí sekce poskytuje pohled na předměty, pouze s tím rozdílem, že neukazuje jejich propojení na obory práva, ale materiály, které jsou k nim přidružené.
\end{description}

Veškerý obsah stránek je pokryt verzováním a je možné ukládat tzv. revize obsahu, mezi kterými pak lze zobrazit výčet změn od předchozí verze. K editaci slouží \gls{wysiwyg} editor \emph{CKeditor}, poskytující uživatelům přívětivé rozhraní s rozšířenými možnostmi úpravy textu a prvky specificky vytvořenými pro použití s \gls{cms} Drupal. Pro prohlížení souborů uložených na serveru ve specifické složce se využívá modulu \emph{IMCE}.

\begin{figure}[htp]
  \includegraphics[width=12cm]{img/workbench-crop.pdf}
  \caption{Prostředí pracovního stolu pro uživatele s právy moderátora.}
  \label{fig:workbench}
\end{figure}  

\section{Výkonové optimalizace}
\emph{Drupal} je všeobecně znám pro svou vysokou náročnost na výpočetní výkon serveru, což je způsobeno jeho rozšiřitelností. Z tohoto důvodu je nutné používat pokročilé techniky ukládání a načítání z vyrovnávací paměti\cite{high-performance-drupal}. Od verze 7 je v jádru implementováno hned několik přístupů ke zvýšení výkonu:

\begin{description}
  \item[Registry] \hfill \\
  Protože mnoho částí systému umožňuje přepisování výstupu za pomocí jmenných konvencí, jsou soubory prohledávány pouze jednou a informace o dostupných funkcích jsou uloženy v databázi. Kvůli tomuto přístupu je nutné přebudovat registry pokaždé, když je přidán nový soubor, například typu \texttt{.tpl.php}. K této operaci slouží modul \emph{Registry Rebuild}\footnote{https://drupal.org/project/registry\_rebuild}. Inicializace registrů nejde vypnout a pouze v nastavení některých témat vzhledu a za pomoci modulu \emph{Devel} lze nastavit, aby se přebudovávaly při každém načtení stránky. S tím však značně klesá výkon.
  \item[Vyrovnávací paměť] \hfill \\
  Za pomocí Cache \gls{api} je možné využívat vyrovnávací paměti typicky uložené v databázi. V jádru je definováno několik tabulek pro ukládání a moduly mohou jednoduše definovat další (například modul \emph{Views} a tabulka \emph{cache\_views}). Data z této vyrovnávací paměti jsou načítáná pouze v případě, kdy je povolena nastavení výkonu systému. V konfiguračním souboru je možné definovat umístění jednotlivých vyrovnávacích pamětí - další umístění mohou být přidávána rozšiřujícími moduly (například Memcache, APC, \dots).
  \item[Agregace] \hfill \\
  Při načítání obsahu stránek je pro každý soubor vytvořen nový požadavek, který je zaslán serveru. Pro snížení náročnosti komunikace je vhodné snížit počet souborů, čehož je typicky dosáhnuto agregací JavaScriptových a \gls{css} souborů (v případě, že je obsah poskytován z více domén, je vhodné ponechat alespoň několik výsledných souborů, které mohou být stahovány paralelně)\cite{website:drupal:optimizing}. Základní funkcionalita je rozšířena modulem \emph{Advanced Aggregation}\footnote{https://drupal.org/project/advagg}.
  \item[Statická vyrovnávací paměť] \hfill \\
  Nepřihlášeným uživatelům, pro které stránky vypadají vždy stejně, je možné posílat výsledný \emph{HTML} kód z vyrovnávací paměti. Stránka však nemůže obsahovat žádný dynamický obsah. Tento problém, zároveň se servírováním stránek přihlášeným uživatelům, řeší modul \emph{Authenticated Cache}\footnote{https://drupal.org/project/authcache}, který namísto dynamicky generované části stránky vytvoří pouze kontejner, do kterého je výsledný obsah přidán za pomoci \emph{AJAX} volání. Uživateli je stránka zobrazena prakticky okamžitě a dynamické prvky se objeví řádově rychleji, neboť pro tento způsob volání je využit okleštěný inicializační skript \emph{Drupalu}.
\end{description}
