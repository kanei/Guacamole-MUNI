\chapter{Analýza}
\label{chap:analyza}
Projekt byl od začátku koncipován jako aktualizace stávajícícho řešení, do kterého budou promítnuty zkušenosti a znalosti nabyté za dobu jeho fungování, zkombinované s moderními technologiemi, které nebyly v době vývoje předešlé verze dostupné. Návrh klade důraz na minimalizaci možných uživatelských chyb a jednoduchost a názornost vytváření obsahu. Těchto cílů dosahuje za pomoci definování vztahů mezi obsahem a omezením možností úprav automatickým generováním prvků webu, jakými jsou například seznamy a tabulky zobrazující propojení elementů.

\section{Existující řešení}
Při analýze bylo nutné zohlednit stávající řešení a možnosti jeho rozšíření. Existující portál využíval \emph{Drupal} v 6. verzi, který je již značně zastaralý a jeho rozšiřitelnost omezená, či neadekvátně komplikovaná. Řešení bylo rozšířeno řadou volně dostupných modulů, které dohromady poskytovaly komplexní strukturu nabídky menu vytvořenou namíru požadavkům fakulty. Struktura byla navržena v době, kdy nebylo známo, že systém bude pro zobrazování zákonů využívat externí software ASPI, proto bylo počítáno s možností zobrazení zákonů přímo v systému. 

Autentizace na server probíhala pouze za pomocí jednotného hesla, které bylo měněno ve čtvrtletních cyklech. Heslo nebylo nijak vázáno na jednotlivé uživatele a tento způsob nedovoloval téměř žádnou kontrolu nad uživateli přistupujícími k obsahu - po získání hesla byl schopen na server přistupovat kdokoliv a zabránění přístupu bylo možné pouze na bázi IP adresy uživatele. 

\subsection*{Struktura menu}

Navigace systému se nachází v levé části stránky a obsahuje v jediné stromové struktuře veškeré odkazy na obsah portálu.

\begin{itemize}
  \item \textbf{[název předmětu 1]} \hfill \\
    rozbalí/sbalí menu
  \begin{itemize}
    \item \textbf{[název předmětu 1]} \hfill \\
      odkaz na stránku předmětu, obsahující pouze název a informace o spolufinancování \gls{eu}
    \item \textbf{Studijní text} \hfill \\
      rozbalí/sbalí menu
      \begin{itemize}
        \item \textbf{Studijní text} \hfill \\
          odkaz na stránku studijní text, která je typicky prázdná
        \item \textbf{Úvod} \hfill \\
          odkaz na stránku obsahující typicky úvodní text k danému předmětu
        \item \textbf{[studijní text 1]} \hfill \\
          jeden z odkazů na stránku s obsahem přednášek, či jinak tématicky oddělených informací k předmětu
        \item \textbf{[studijní text 2]}
      \end{itemize}
    \item \textbf{Ostatní studijní materiály} \hfill \\
      tento a následující již nejsou pevně dané a vyskytují se nepravidelně pouze u některých předmětů
    \item \textbf{Prezentace}
    \item \textbf{Právní předpisy}            
    \item \textbf{Judikáty}
    \item \textbf{Procvičování}        
  \end{itemize}
  \item \textbf{[název předmětu 2]}
  \item \textbf{[název předmětu 3]}
\end{itemize}

Z výše popsané struktury vyplývá, že její vlastnosti nesplňují požadavky moderních informačních systémů. Odkazy mají nekonzistentní chování - buď se rozbalí pod-menu s dalšími odkazy, nebo prohlížeč přejde na novou stránku. Uživatel proto nemůže jednoduše vědět, jaké chování od své akce očekávat. Struktura obsahuje duplicity (název předmětu a odkaz na studijní text se opakují a v obou případech mají jiný význam). Dále není určená společná struktura všech předmětů a tím může lehce dojít ke zmatení uživatelů, kteří musejí pokaždé informace hledat na jiném místě (stránky \emph{Ostatní studijní materiály}, \emph{Prezentace}, \dots). Některé stránky také místo agregace užitečných informací obsahují buď pouze banner, nebo jsou úplně prázdné (úvodní stránka předmětu) v místech.

\section{Použité technologie}
\label{sec:technologies}

Základními technologiemi jsou \emph{Drupal} a \emph{Guacamole}, které definují vyžití dalších technologií. Základem jsou programovaí jazyky \emph{PHP}, \emph{JAVA} a \emph{JavaScript} a výstup je realizován pomocí \emph{HTML} (v.4 a v.5) a \emph{CSS} (v.2 a v.3). Ačkoliv bylo využito velké množství modulů, zmíněny budou pouze pouze ty, které měly zásadní vliv na funkcionalitu nebo proces vývoje.

\subsection{Jádro CMS a přidružené technologie}

\subsubsection*{\textbf{Drupal} \hfill \emph{http://drupal.org}} 
\label{subsec:drupal}
V době psaní této práce byl \emph{Drupal} světově třetím nejrozšířenějším \gls{cms}\cite{website:cms-market-share}. Založený na jazyce PHP, klade důraz na vývojáře a na možnosti úpravy stránek, proto často bývá označován za \gls{framework}. Konkurenční \gls{cms} \emph{WordPress} cílí na uživatelskou jednoduchost a většina stránek na něm postavných je lehce rozpoznatelná, zatímco \emph{Drupal} je možné změnit od základu a jeho \gls{api} pro tvorbu modulů poskytuje komplexní možnost úpravy. Minimálním požadavkem pro spuštění je \emph{PHP} verze 5.3 a i když některé jeho části jsou již implementovány objektově, celkově převládají čísté funkce s množstvím propriétárních principů řešení. Mezi ty patří například hook\_api, poskytující přípojné body pro další moduly a tím rozšiřitelnost částí systému, nebo systém vzorů (templates), které pomocí speciální jmenné konvence umožňují měnit výpis html kódu prvků webu (viz. kapitola \ref{sec:tema-vzhledu}).

\emph{Drupal} podporuje využití tzv. instalačních profilů. Jedná se o speciální moduly neposkytující samy o sobě žádnou funkcionalitu, ale sdružující informace o potřebných modulech a jejich iniciálním nastavení. To lze využít pro specifické distribuce, například vícejazyčná instalace, řešení pro blog, řešení orientované na výkon a podobně.

\subsubsection*{\textbf{Kerberos} (Protocol) \hfill \emph{http://web.mit.edu/kerberos}}
Vyvinut na \emph{Massachusettském technologickém institutu (MIT)}, protokol \emph{Kerberos} poskytuje autentizační funkcionalitu za pomocí symetrické kryptografie (jediný klíč je použit pro šifrování i dešifrování). Autentizace je provedena za pomoci důvěryhodného třetího serveru (v tomto případě \gls{is} Masarykovy Univerzity).

\subsubsection*{\textbf{Omega} (Drupal Theme) \hfill \emph{http://drupal.org/project/omega}}
\label{subsec:omega}
Pro \emph{Drupal} existuje nespočet témat vzhledu, které se starají o formátování výstupu html kódu a s ním spojených kaskádových stylů \gls{css}. Projekt \emph{Omega} je zaměřen na \gls{responsive} a poskytuje implementaci stylů pro různá zobrazovací zařízení. Zatímco třetí verze obsahuje komplexní administrační rozhraní, ve kterém lze nastavit rozmístění a poměry mezi bloky stránky, přístup čtvrté verze se zaměřil na implementaci v kódu. Poměry mezi jednotlivými bloky jsou zapsány pomocí nástroje \emph{SUSY} a pomocí knihovny \emph{Breakpoints}\footnote{http://xoxco.com/projects/code/breakpoints/} a vlastnosti \gls{mediaqueries} se mezi sebou přepínají.

\subsubsection*{\textbf{SASS} (Syntactically Awesome Style Sheets) \hfill \emph{http://sass-lang.com/}}
\label{subsec:sass}
Díky tématu vzhledu \emph{Omega} se nabídla možnost využití nástroje \emph{SASS} pro generování popis vzhledu \emph{HTML} elementů. Namísto přímého použití \gls{css} je kód napsán ve formátu \emph{SASS} a poté zkompilován do \gls{css}. Tento přístup přináší bezproblémovou kompatibilitu se všemi prohlížeči doplněnou o rozšířené možnosti definice pravidel, jako je využití proměnných, vnoření pravidel, in-line import a další. Kromě výrazného vylepšení čitelnosti a možnosti seskupováni pravidel bez zvyšování zátěže na přenos dat tak lehce lze dosáhnout i snížení programátorské náročnosti a zvýšení efektivity.

\subsubsection*{\textbf{COMPASS} \hfill \emph{http://compass-style.org/}}
\label{subsec:compass}
Soubory \emph{SASS} musí být kompilovány do \gls{css} pro zobrazení prohlížečem a protože manuální kompilace vyžaduje opakované spouštění příkazů a je tím pádem náchylná k opomenutí, nabízí se využití programu \emph{COMPASS}. Přestože poskytuje širokou škálu funkcionality určené pro ulehčení práce designérům, pro potřeby tohoto projektu byla využita pouze automatická konverze z \emph{SASS} do \gls{css} dle definovaných pravidel za pomocí příkazu \texttt{\$ compass watch}. Po spuštění démona jsou kontrolovány všechny v souborech ve složce a automaticky regenerovány výstupní \gls{css} soubory pro načtení prohlížečem.

\subsubsection*{\textbf{SUSY} \hfill \emph{http://susy.oddbird.net/}}
\label{subsec:susy}
Další technologií využitou v tématu vzhledu je \emph{SUSY} - implementace responsivní mřížky pro \emph{COMPASS}. Za pomocí jednoduchých pravidel lze nadefinovat rozdílné rozvržení stránky v závislosti na rozlišení zobrazovacího zařízení. Například monitoru počítače s rozlišením vyšším než 1024 bodů můžeme postranní panel zobrazit nalevo, zatímco na mobilním zařízením můžeme text zmenšit a zobrazit v horní části stránky spolu s vypuštěním některých nedůležitých bloků. Celá stránka je může být rozdělena na počet sloupců, které mohou být dynamicky vyplňovány a pravidla lze velmi jednoduše zapisovat bez nutnosti řešení problémů s kompatibilitou mezi prohlížeči.

\subsection{Technologie připojení ke vzdálené ploše}

\subsubsection*{\textbf{Guacamole} \hfill \emph{http://guac-dev.org/}}
\label{subsec:guacamole}
Jak je popsáno na stránkách této knihovny, jedná se o potvrzení myšlenky připojení ke vzdálené ploše skrze webový prohlížeč. Namísto potřeby instalace speciálního klienta je možné používat jakýkoliv prohlížeč podporující některé z technologií \emph{HTML5} (např. canvas). Připojení probíhá skrze proxy server implementovaný v jazyce \emph{C}, který zprostředkovává připojení a komunikaci. Ke klientovi putují již jen obrazová data a povely, obojí zakódováno v proprietárním protokolu, který umožňuje snížit odezvu obrazu a dekódovat jej v průběhu přijímání dat. Postup komunikace mezi klientským prohlížečem a vzdálenou plochou je blíže vyobrazen na diagramu \ref{fig:arch_core}.

\begin{figure}[htp] 
  \centering{\includegraphics[width=12cm]{img/architecture-core-crop.pdf}}
  \caption{Architektura komunikace nástroje Guacamole skrze vzdálenou plochu}
  \label{fig:arch_core}
\end{figure}  

Propojení mezi webovým prohlížečem a proxy serverem je provedeno za pomocí \gls{ajax} požadavků v čistém jazyce \emph{JavaScript}, zatímco propojení z proxy dále podporuje protokoly \emph{SSH}, \emph{VNC} a \gls{rdp}.

\subsubsection*{\textbf{CORS} \hfill \emph{http://www.w3.org/TR/cors/}} 
Kvůli bezpečnostním rizikům je \emph{JavaScriptu} běžícímu v prohlížeči zakázáno přistupovat k jiným doménám, tento přístup se nazývá \emph{Same-origin policy} a slouží k zamezení podvodných stránek a zvýšení internetové bezpečnosti. Veškeré požadavky směřující na jinou doménu tak musejí být vykonávány přímo ze serveru a prohlížeči zasílány pomocí protokolu \gls{ajax}. Tento přístup je často zbytečně náročný a v dnešní době bývá nutné tato omezení obejít. K tomu slouží \gls{cors}~\cite{website:cors}, kdy za pomocí \emph{HTML} hlaviček přidaných do komumnikace mezi serverem a prohlížečem a nastavení serveru je umožněna komunikace s jinými doménami. Pokud server odešle zpět hlavičku \texttt{Access-Control-Allow-Origin: [doména]}, prohlížeč ji zanalyzuje a případně umožní komunikaci. K tomuto je nutné využít \emph{XMLHttpRequest} (\emph{XDomainRequest} v případě \gls{ie}) s attributem \texttt{WithCredentials}, o zbytek se postará prohlížeč se serverem a komunikace funguje stejně jako v případě klasických požadavků.

\subsubsection*{\textbf{ASPI} \hfill \emph{http://www.systemaspi.cz/}}
Systém \emph{ASPI} poskytuje jeho uživatelům komplexní právní informace z prostředí České republiky. Velký důraz je kladen na možnost jednoduchého vyhledávání a verzování obsahu, stejně jako dostupnost skrze hlavní operační systémy. \emph{ASPI} funguje jako aplikace instalovaná na klientském prostředí, ale nabízí také možnost vytvoření serverů dostupných k připojení skrze vzdálenou plochu, čehož je využito v případě tohoto projektu.

\subsection{Podpora nasazení a vývoje projektu}

\subsubsection*{\textbf{Drush} \hfill \emph{http://github.com/drush-ops/drush}}
\label{subsec:drush}
Pro ulehčení administračních úkonů nad instalací \emph{Drupalu} byl komunitou vyvinut program \emph{Drush}, poskytující textové administrační rozhraní nad Drupalem v terminálové konzoli. V základu jsou umožněny operace nad moduly a tématy vzhledu a každý z modulů může pomocí implementace funkcí v souboru \texttt{[název].drush.inc} zoršířit rozhraní vlastními příkazy.

\subsubsection*{\textbf{Phing} \hfill \emph{http://www.phing.info/}}
\label{subsec:phing}
\emph{Phing} je nástroj určený ulehčení \gls{deployment} a nastavení \emph{PHP} projektů, fungující na bázi \emph{XML} konfiguračního souboru. Je založený na podobné myšlence jako \emph{Apache ANT}, který poskytuje podobnou funkcionalitu pro jazyk \emph{JAVA}. V konfiguračním souboru lze definovat pravidla, proměnné a závislosti vedoucí k \gls{deployment} projektu, kontrolám kódu, či automatizaci jakýchkoliv dalších akcí, potřebných k správnému chodu aplikace.

\subsubsection*{\textbf{DrushTask} \hfill \emph{https://drupal.org/project/phingdrushtask}}
Tento projekt využívá možnosti vývoje vlastních operací (task) v nástroji \emph{Phing}. Za pomoci příkazu \texttt{TaskdefTask}\cite{website:phing-user-guide} je definovaná operace \texttt{<drush>}, poskytující rozhraní k programu \emph{Drush}. Tím je umožněno automatizovat většinu administrátorských činností nad projektem a případně implementovat automatickou analýzu kódu a testování pomocí \gls{ci}.

\subsubsection*{\textbf{Maven} \hfill \emph{http://maven.apache.org/}}
Podobně jako \emph{Phing} pro {PHP}, \emph{Maven} poskytuje automatizaci \gls{deployment} \emph{JAVA} projektů (převážně). Jako základ také používá \emph{Apache ANT}, který rozšiřuje a specializuje se na možnost znovuvyužití skriptů pro více projektů. 

\subsubsection*{\textbf{Team City} \hfill \emph{http://www.jetbrains.com/teamcity/}}
Podobně jako \gls{opensource} komunitou často využívaný \emph{Jenkins} či \emph{Hudson}, \emph{Team City} je implementaci \gls{ci} serveru. V případě tohoto řešení jde o čistě komerční produkt, který je ale pro malé projekty dostupný zcela zdarma (je omezen na počet projektů). \emph{Team City} umožňuje komplexní nastavení projektů a jednotlivých úkonů, kontrolujících zdrojový kód, nebo provádějících operace \gls{deployment} na produkční prostředí. Je také poskytována integrace s většinou moderních \gls{cvs}.

\subsubsection*{\textbf{GitHub} \hfill \emph{https://github.com/}}
\label{subsec:github}
Pro udržování historie kódu a pro spolupráci více autorů byla vybrána \gls{cvs} technologie \emph{Git} a online portál \emph{GitHub}, který poskytuje jednoduchou instalaci a online správu uložiště. Všechny úpravy mohou být jednoduše zobrazeny přímo v prohlížečí a provázány s úkoly vytvořenými v úkolovém managementu na stejném internetovém portálu. 

\section{Struktura portálu v prostředí Drupalu}

Při vývoji informačních systémů jsou typicky data ukládány v databázi. To stejné platí i pro Drupal, pouze s tím rozdílem, že poskytuje administrační rozhraní schopné dynamicky vytvářet úložiště pro data. Namísto vytváření nové tabulky jsou uživatelé schopni definovat typ obsahu, který navíc může dědit určité vlastnosti. Základním typem je Entita, poskytující možnost ukládání základních informací. Entita označuje abstraktní pohled na data a nemůže být instancializována, namísto toho musí nejdříve být definován typ entity, na jehož základě až může být vytvořen typ obsahu. Pro ilustraci, v jádře drupalu existuje typ entity Uzel (Node), definující obsah s vlastní \gls{url} adresou a titulkem. Ten však nemůže být vytvořen přímo a administrátor musí definovat typ obsahu (Bundle), který pojmenuje \emph{Stránka} (Page). Tu je již možno vytvářet a následně k ní přistupovat. \emph{Stránka} je tedy uzlem a entitou. Toto však neplatí pro všechny typy obsahu.  soubory (File) a slovníky (Vocabulary) lze vytvářet přímo a nelze definovat jejich potomky\cite{drupal-entities}. Seznam všech základních entit je zobrazen v tabulce~\ref{tab:typy-entit}.

\begin{table}
  \caption{Základní typy entit v Drupalu}
  \label{tab:typy-entit}
  \begin{tabular}{ | p{3cm} | l | c | c | }
    \hline 
    Typ entity & Strojový název & Dostupnost polí & Rozšiřitelnost \\ \hline 
    Komentář & comment & \checkmark & \checkmark \\ \hline 
    Soubor & file &  & \\ \hline 
    Slovník & vocabulary &  & \\ \hline 
    Uzel & node & \checkmark & \checkmark \\ \hline 
    Uživatel & user & \checkmark & \checkmark \\ \hline 
    Záznam slovníku & term & \checkmark & \checkmark \\ \hline             
  \end{tabular}
\end{table}

Pokud daný typ entity poskytuje možnost ukládání polí, je možné mu přiřadit pole, které si lze předststavit v analogii databáze jako sloupce tabulky. Každé pole je však reprezentováno jako tabulka v databázi a tak je při přístupu k jedné entitě v Drupalu ve skutečnosti nutno přistoupit do tolika tabulek, kolik má daná entita přiřazených polí. Zároveň jdou pole využívat skrze více typů entit a tak mohou být pole sdílena skrze celý systém, například typ obsahu \emph{Stránka} a typ obsahu \emph{Novinka} mohou oba poskytovat pole \emph{Obrázek}, které bude v databázi ukládáno do jediné tabulky \emph{field\_obrazek}. Instance obou typů obsahu dostanou přiděleno unikátní číslo v rámci všech entit, které určuje její data v databázi. Ne všechna pole musejí být zobrazena na stránce entity, nebo může pro jejich zobrazení být využito speciálních formátovacích technik, jako tabulek, či jiných výpisů. Všechna tato nastavení lze opět definovat buď v kódu, nebo skrze administrační rozhraní \emph{Drupalu}.

\subsection{Typy polí použitelné pro vlastnosti typů obsahu}
V drupalu existuje několik základních typů polí, které fungují podobně jako datové typy v klasických programovacích jazycích. Narozdíl od jednoduchých datových typů je ke každému poli přiřazeno jedno či více formátování výsledného \emph{HTML} kódu a zpravidla poskytují rozšiřující nastavení. Formátování jde dále změnit pomocí přepsání základních formátovácích funkcí, či vytvoření a přiřazení funkcí nových.

\begin{description}
  \item[Text] Poskytuje pole pro text s omezenou délkou bez možnosti vkládání html značek. Jeho obsah je zadáván pouze v jediném řádku a poskytuje pouze limitované možnosti nastavení, kterými je délka vstupního pole, či výchozí hodnota. 
  
  \item[Dlouhý text se souhrnem] Narozdíl od typu pole \emph{Text} může v tomto případě být text neomezeně dlouhý a obsahovat značky html kódu. Ty mohou být vkládány buď ručně a nebo být automaticky formátovány pomocí \gls{wysiwyg} editoru. Zároveň je pro text možné vyplnit souhrn, který je ve specifických případech zobrazit - jedná se například o seznam aktualit na úvodní stránce. Pokud souhrn není vyplněn, typicky se zobrazí prvních několik set znaků obsahu.
  
  \item[Hodnota slovníku] V nastavení pole lze vybrat jeden z existujících slovníků (více o slovnících v kapitole \ref{sec:slovniky}). Jeho hodnoty jsou vypsány v jednom z dostupných formátů (radio/zaškrtávací tlačítka) a dle nastavení lze vybrat jen jedna, nebo více možností.
  
  \item[Logická hodnota] Poskytuje pole pro logickou hodnotu - ano / ne. Pro hodnoty lze vybrat jednu z typicky použítelných nápisů (ano/ne, pravda/nepravda, \checkmark/x), nebo definovat nápisy vlastní. 

  \item[URL adresa] Kromě samotné adresy poskytuje možnost uložení dalších parametrů, podobně jako element <a> jazyka html. Mezi tyto parametry patří např. titulek, cíl a podobně.

  \item[Média] Pro vkládání médií do systému je využit modul Media (kapitola~\ref{subsec:media}), který přidává komplexní řešení pro vkládání multimediálního obsahu. Je možné vkládat dokumenty, soubory pdf, zvuk, obraz, či video a všechny tyto typy obsahu jsou uživately dostupné v jednotném formátu.
  
  \item[Váha] I když by se pro určení váhy produktu dalo využít jednoduchého číselného pole, jeden z rozšiřujících modulů poskytuje speciálně využitelné pole pro výběr hodnoty a také pohledy na obsah a řadící funkce, které celou implementaci značně ulehčují.
  
  \item[Odkaz na entitu] Důležitou vlastností většiny informačních systémů je možnost propojení entit. Pro modelování těchto vztahů existuje typ pole \emph{odkaz na entitu}, u kterého je možné vybrat typy entity, na které je entita navázána a způsob propojení. Za pomocí extra modulů lze modelovat místo jednoduché asociace kompozici (při smazání rodiče jsou smazány i všechny propojené moduly). Lze také definovat, zda je možné pouze vytvářet nové podentity, či přidávat ty již existující.
\end{description}
