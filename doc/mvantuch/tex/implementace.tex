\chapter{Implementace portálu v Drupalu}
\label{chap:implementace-drupal}

Pro převedení stávajícího řešení na novou verzi bylo nutné vykonat několik kroků. Nejprve byla data z produkčního prostředí přesunuta na testovací server, kde bylo řešení zanalyzováno co se týče implementační stránky, neboť bylo možné k portálu přistupovat s administrátorským účtěm bez jakýchkoliv omezení. Jakmile byly stránky zanalyzovány, byl započat proces jejich aktualizace na novou verzi Drupalu při zachování veškerého obsahu. Tento postup se ukázal jako kontraproduktivní, neboť i když byl obsah stránek převeden vpořádku, bylo potřeba jej převést na kompletně odlišnou strukturu spolu s jinými typy obsahu a jinými cílovými poli. Proto bylo rozhodnuto, že data budou přesunuta ručně.

\section{Aktualizace Drupal 6 na Drupal 7}
Sedmá verze Drupalu komplexně přepracovala celý kód redakčního systému a zohlednila v té době relevantní požadavky na moderní \gls{cms}. Mnoho modulů bylo přesunuto přímo do jádra systému - například podpora vytváření vlastních polí obsahu, pro což byl v Drupalu 6 potřeba modul CCK (Content Creation Kit). Množství změn je jednoduše vidět i ve velikosti zdrojových souborů, kdy verze 7 obsahuje 12.8~MB zdrojového kódu oproti 3,7~MB u verze 6.

Kromě funkcionality přidané do jádra, které by bylo možné dosáhnout i za pomocí modulů v předchozí verzi, byly přepracovány i mnohé části funkcionality jádra systému~\cite{website:drupal-comparison}. Pro toto řešení jsou důležitá následující vylepšení:

\begin{itemize}
  \item podpora kešování za pomocí reverzní proxy (např. Varnish)
  \item Entity API - unifikace elementů stránek a API ulehčující vytváření nových typů obsahu, třídy pro ulehčení práce~\cite{drupal-entities}
  \item možnost přidávat pole ke všem entitám (uživatelé, slovníky, ...)
  \item podpora různých databází se stejným zdrojovým kódem v PHP (abstrakce nad dotazy do databáze)
  \item zlepšená podpora AJAX volání
  \item celkově propracovanější kód (vhodný pro vývoj dalších modulů)
\end{itemize}

V době práce na informačním systému již byl dostupný i Drupal verze 8, ale nebyl ještě plně stabilní. Některé jeho vlastnosti však byly zpětně implementovány jako modul pro Drupal 7 a mou snahou bylo identifikovat jakékoliv takovéto moduly (například NavBar, který poskytuje responsivní administrační panel, filtrace modulů, či administrační téme Ember postavené na SASS). 

\begin{table}
  \caption{Porovnání verzí modulů mezi Drupalem 6 a 7}
  \begin{tabular}{ | p{5cm} | p{2.5cm} | p{2.5cm} | c | }
    \hline 
    Jméno modulu & Drupal 6 & Drupal 7 & stav  \\ \hline 
    Backup and Migrate & 2.7 & 2.7 & \checkmark \\ \hline
    Colorbox & 1.6 & 2.4 & \checkmark \\ \hline
    CCK & 2.9 & jádro & \checkmark \\ \hline
    Custom Breadcumbs & 1.5 & 1.x-alpha3 & \\ \hline
    DB Maintenance & 1.4 & 1.1 & \checkmark \\ \hline
    DHTML Menu & 3.5 & 1.0-beta1 & \\ \hline 
    Email Field & 1.4 & 1.2 & \checkmark \\ \hline
    File (Field Paths & 1.5 & 1.0-beta4 & \\ \hline
    FileField & 3.11 & jádro & \checkmark \\ \hline
    Front Page & 1.3 & 2.4 & \checkmark \\ \hline
    ImageAPI & 3.11 & jádro & \checkmark \\ \hline
    ImageCache & 2.0-rc1 & jádro & \checkmark \\ \hline
    IMCE & 2.5 & 1.7 & \checkmark \\ \hline
    IMCE Wysiwyg bridge & 1.1 & 1.0 & \checkmark \\ \hline
    jCarousel & 2.6 & 2.6 & \checkmark \\ \hline
    Link & 2.10 & 1.1 & \checkmark \\ \hline
    Localization Update & 2.10 & 1.1 & \checkmark \\ \hline
    Menu Attributes & 2.0-beta1 & 1.0-rc2 & \checkmark \\ \hline
    Menu Block & 2.4 & 2.3 & \checkmark \\ \hline
    Pathauto & 1.6 & 1.2 & \checkmark \\ \hline
    Taxonomy Breadcrumb & 1.1 & - & \\ \hline
    Token & 1.19 & 1.5 & \checkmark \\ \hline
    Transliteration & 3.1 & 3.1 & \checkmark \\ \hline
    Views & 2.16 & 3.7 & \\ \hline
    Views Search & 1.0 & - & \\ \hline
    WYSIWYG & 2.4 & 2.2 & \checkmark \\ \hline
  \end{tabular}
\end{table}

Funkcionalita některých z modulů byla v sedmé verzi přesunuta přímo do jádra Drupalu a proto nebylo jejich využití již potřebné - byly využity jejich ekvivalenty z jádra. Kromě těchto byly odebrány i další moduly, kdy některé nejsou pro novou verzi vůbec dostupné. Dále byly odstraněny moduly, jejichž využití již postrádalo smysl. Z důvodu změny a zjednodušení struktury menu již nebylo potřeba mnoho modulů, které umožňovaly složitá menu zobrazovat uživatelům. Zároveň bylo odstraněno několik modulů, které nebyly na stránkách vůbec využity a zbytečně zpomalovaly běh portálu. Následuje seznam těchto modulů s jejich krátkým popisem.

\subsection{Odebrané moduly (nedostupné)}
Z důvodu převážného vyvíjení Drupalu komunitou nebyly všechny moduly dostupné pro Drupal 6 vyvinuty i pro Drupal 7 - většinou z důvodu nepotřebnosti daných modulů, kdy u většiny užitečných modulů nová verze existuje.

\subsubsection*{Views Search \hfill \emph{https://drupal.org/project/views\_search}}
Rozšiřuje funkcionalitu Views a jejich obsahových filtrů.

\subsection{Odebrané moduly (definující strukturu předchozího řešení)}
Předchozí struktura webu byla v mnohém odlišná a k její implementaci bylo využito několik modulů, které mohly být pro nepotřebnost vypnuty a odinstalovány. 

\subsubsection*{Front Page \hfill \emph{https://drupal.org/project/front}}
Slouží k nastavení výchozí stránky pro různé uživatelské role. Z důvodu plánování kompletně nové implementace domovské stránky byla tato funkcionalita prozatím odstraněna a může být případně přidána v čase implementace dané části portálu.

\subsubsection*{Menu Attributes \hfill \emph{https://drupal.org/project/menu\_attributes}}
Umožňuje přiřazení attributů k položkám menu. Ty pak mohou být použity buď pro stylování, nebo navázány na JavaScript.

\subsubsection*{Taxonomy Breadcrumb \hfill \emph{https://drupal.org/project/taxonomy\_breadcrumb}}
Generuje drobečkovou navigaci za pomoci slovníků definovaných v systému a jejich hierarchie. Hierarchie tím pádem může být velmi jendoduše upravena - toto však již není po úpravě systému potřeba s ohledem na zjednodušenou strukturu.

\subsubsection*{Custom Breadcrumbs \hfill \emph{https://drupal.org/project/custom\_breadcrumbs}}
Umožňuje detailní nastavení drobečkové navigace.

\subsubsection*{DHTML Menu \hfill \emph{https://drupal.org/project/dhtml\_menu}}
Mění funkcionalitu menu - při prvním kliknutí se namísto přesunutí na danou stránku pouze rozbalí menu s potomky daného odkazu a až při druhém kliknutí se přejde na danou stránku - tato funkcionalita zřejmě nebyla využívána.

\subsection{Nevyužité moduly}
Několik modulů sice bylo povoleno, ale nebyly nikde v systému využity a proto byly odstraněny.

\subsubsection*{jCarousel \hfill \emph{https://drupal.org/project/jcarousel}}
Slouží k propojení se stejnojmennou JavaScriptovou knihovnou poskytující zobrazení obrázků v ovladatelném ,,kolotoči''. S ohledem na aktuální absenci jakékoliv grafiky byl modul shledán nepotřebným.
 
\subsubsection*{Colorbox \hfill \emph{https://drupal.org/project/colorbox}}
Podobně jako jCarousel slouží k propojení s knihovnou určenou ke zobrazování obrázků a ani tento modul nebyl aktivně využíván.

\section{Nově přidané moduly}
\label{sec:nove-moduly}
Do projektu byly zahrnuty další moduly, některé z nich rozšiřující funkcionalitu řešení, jiné za účelem zjednodušení vývoje a údržby.

\subsubsection*{Context \hfill \emph{https://drupal.org/project/context}} 
\label{subsec:context}
Modul Context umožňuje definovat reakce v závislosti na splněných podmínkách prostředí portálu. Pro každou podmínku, kterou může být například přihlášená role uživatele, aktuální stránka či jazyk, lze nadefinovat změny v portálu. Ať se jedná o nastavení drobečkové navigace, změnu obsahu bloku a podobně. Podmínky lze mezi sebou vzájemně kombinovat a uzavírat do logických výrazů, stejně jako lze pro jeden logický výraz provézt více operací. Pro potřeby tohoto projektu je využíván hlavně podmodul Context Layouts a další modul Context Omega\footnote{https://drupal.org/project/conext\_omega}, které dohromady umožňují změnu rozložení stránek.

\subsubsection*{Features \hfill \emph{https://drupal.org/project/features}} 
\label{subsec:features}
Jak je popsáno v kapitole~\ref{chap:vyvoj}, Features slouží k ulehčení vývoje a správy nastavení stránek postavených na Drupalu. Základní myšlenkou je třívrstvé rozvržení systému, v první úrovni je jádro a jeho moduly, ve druhé rozšiřující moduly (tzv. Contrib) a ve třetí jsou rysy (Features), které v sobě zapouzdřují samotné nastavení stránek. Ačkoliv se jedná také o moduly, jsou generovány systémem na stránce \texttt{admin/structure/features}. Zde je možné buď vytvořit nový rys, či spravovat ty stávající. Ty mohou být v několika stavech závisejících na aktuálním nastavení systému. Pokud bylo v sytému něco změněno od chvíle, kdy byl rys vygenerován a povolen (případně pouze povolen, pokud byl přenesen z jiného serveru), je možné dané změny buď vrátit ke stavu definovanému v rysu, nebo rys aktualizovat na novou verzi.

\subsubsection*{Features Extra\hfill \emph{https://drupal.org/project/features\_extra}} 
Jelikož modul Features v sobě implementuje pouze základní funkcionalitu exportu typů obsahu a poskytuje \gls{api} pro další moduly, některé prvky webu nejsou v základu exportovatelné. Tento modul možnosti exportu rozšiřuje o několik dalších možností, ze kterých pro tento projekt byla důležitá možnost exportovat nastavení umístění bloků.

\subsubsection*{@font-your-face\hfill \emph{https://drupal.org/project/fontyourface}} 
Pro využití řezů písem třetích stran je možné postupovat několika způsoby - buď nahrát soubor obsahující písmo přímo na server, nebo se odkazovat na server třetí strany, který daný font poskytuje (např. Google Fonts API). Modul @font-your-face unifikuje tyto přístupy a poskytuje jednoduché rozhraní pro výběr typů a řezů písma a jejich použití na stránkách. Kód lze pak jednoduše vložit do souboru definujícího téma vzhledu - například řádek \texttt{fonts[google\_fonts\_api][] = "BenchNine\&subset=latin-ext\#300"} automaticky načte z Google Fonts API font se jménem BenchNine v řezu 300.

\subsubsection*{Entity API\hfill \emph{https://drupal.org/project/entity}}
V předchozí verzi Drupalu bylo možné vytvářet pouze uzly (node), které s sebou nesou nezanedbatelnou složitost v podobě vlastní \gls{url}, revizí a dalších informací (proto například i slovníky jsou vlastně uzly - stránkami). Ve verzi 7 je již možné vytvořit i prakticky čistou entitu, u které lze nastavit všechny vlastnosti, například zda ukládá informace o autorovi a podobně. V základu je však tato implementace poněkud komplikovaná a není nijak lehce dostupná administrátorům, ani vývojářům a proto byl vyvinut modul Entity API, poskytující rozhraní, umožňující jednoduchou definici entity na bázi zdrojového kódu. 

\subsubsection*{Entity Creation Kit (ECK)\hfill \emph{https://drupal.org/project/eck}} 
Zatímco modul Entity API poskytuje možnost definovat entity ve zdrojovém kódu, za pomoci modulu ECK je možno je definovat čistě z administračního rozhraní, čímž je celý proces tvorby značně zjednodušen. Definované entity pak lze exportovat za pomocí modulu Features do kódu modulu a tím zabezpečit jejich verzování a možnost opravy případných chyb.

\subsubsection*{Entity Reference\hfill \emph{https://drupal.org/project/entity\_reference}}
Spolu s modulem Inline Entity Form\footnote{https://drupal.org/project/inline\_entity\_form} poskytuje modul Entity Reference uživatelsky přívětivou možnost k propojení entit mezi sebou. Každá entita (takže i všechny uzly) tak mohou odkazovat na jednu či více entit buď jiného, či stejného typu a mohou být pomocí modulu Inline Entity Form vytvářeny přímo z entity rodičovské. 

\subsubsection*{Media \hfill \emph{https://drupal.org/project/media}}
\gls{cms} systém Drupal v základu poskytuje pouze jednoduchou funkcionalitu práce s obrázky. Zvuk, video i dokumenty nejsou v aktuální verzi (7.x) podporovány vůbec. Zároveň uživatelé nemají mnoho možností operovat se soubory, které nahráli na server a tím pádem jsou často nuceni obsah opětovně nahrávat, čímž docházi k duplikaci. Tyto a mnohé další problémy se snaží řešit projekt Media, poskytující uživatelům srozumitelné rozhraní a unifikaci práce s různými typy multimediálního obsahu a zároveň správu veškerého obsahu nahraného na server. Pro svůj běh vyžaduje modul File Entity\footnote{https://drupal.org/project/file\_entity}, který poskytuje možnosti práce se soubory jako entitami v systému.

\subsubsection*{Views Bulk Operations (VBO) \hfill \emph{https://drupal.org/project/views\_bulk\_operations}}
\label{subsec:vbo}
Modul Views (\emph{https://drupal.org/project/views}) rozšiřuje funkcionalitu CMS Drupal o uživatelsky definovatelné pohledy na data (podobně jako v databázi) a v dnešní době je nainstalován téměř na každé instalaci Drupalu (ve verzi 8 je již obsažen v jádru). Tento modul však není sám schopen poskytovat operace nad více řádky najednou, s čímž pomáhá modul Views Bulk Operations - ke každému řádku je přiřazeno zaškrtávací tlačítko a nad formulářem je zobrazen seznam operací, které lze nad elementy provádět.

\subsubsection*{Workbench \hfill \emph{https://drupal.org/project/workbench}}
\label{subsec:workbench}
Administrace Drupalu je mnohdy kritizována pro její nízkou uživatelskou přívětivost a složitost, čemuž se snaží ulehčit modul Workbench. Přihlášenému uživateli s právem administrace, či správy obsahu je umožněno přistoupit ke stránkám postavených na modulu Views a poskytujících přehled obsahu stránek - ať už se jedná o všeobecně nejnovější obsah, nebo o obsah naposledy upravený daným uživatelem. Tato funkcionalita je dále rozšířena, jak je popsáno v kapitole \ref{chap:vyvoj}.

\subsubsection*{Workbench Media \hfill \emph{https://drupal.org/project/workbench\_media}}
Zatímco modul Workbench poskytuje pohled na obsah stránek, Workbench Media, jak již název nápovídá, přidává propojení s modulem Media a tím pádem pohled na mediální obsah daného uživatele. 

\section{Téma vzhledu a jeho rozložení}
Pro rozvržení stránek bylo využito tématu vzhledu \emph{Omega} ve verzi 4. Narozdíl od předchozích verzí, které poskytovaly uživatelům komplexní administrační rozhraní pro nastavení rozvržení stránky, byla tato možnost z verze 4 odstraněna a nahrazena využitím knihovny \emph{SUSY} (sekce~\ref{sec:technologies}). Veškerá rozvržení jsou proto implementována přímo v kódu a v administračním rozhraní jsou určovány až pozice jednotlivých bloků Drupalu. Každé téma vzhledu v prostředí Drupalu definuje množinu dostupných segmentů stránky, do kterých lze vkládat bloky s vlastním obsahem portálu. 

\begin{figure}[htp] 
  \centering{\includegraphics[width=12cm]{img/page-layout-crop.pdf}}
  \caption{Rozvržení stránky}
  \label{fig:rozvrzeni-stranky}
\end{figure}  

Jak lze vidět na obrázku \ref{fig:rozvrzeni-stranky}, je stránka rozdělena na čtyři hlavní segmenty - hlavičku, postranní panel, obsah a patičku. Pro možnosti budoucího rozšíření je navíc definován segment Postskript, který se nachází ve spodní části obsahové části stránky. Zatímco hlavička i patička zabírají celou dostupnou šíři stránky (960px), centrální segment je rozdělen mezi postranní panel a obsah - poměr mezi těmito segmenty je určen za použití knihovny \emph{SUSY} a media queries a mění se v závislosti na šíři okna prohlížeče~\cite{responsible-drupal}. 

\begin{description}
  \item[Telefon] (velikost okna do 44em) \hfill \\
  Postranní panel zabírá celou šířku a za ním následuje obsah.
  \item[Tablet + standart] (velikost okna mezi 44em a 100em) \hfill \\
  Postranní panel a obsah jsou zobrazeny v poměru 1 : 3.
  \item[Širokoúhlý monitor] (velikost okna nad 100em) \hfill \\
  Postranní panel a obsah jsou zobrazeny v poměru 1 : 4.  
\end{description}

Toto rozložení je vhodné pro využití při klasickém procházení stránkou. S ohledem na využítí \emph{Guacamole} a připojení se na vzdálenou pracovní plochu přímo z portálu a tím pádem za využití Drupalu bylo potřeba zajistit vhodnou strukturu html kódu stránky. Pro oddělení těchto dvou nesourodých prostředí je využito implementace různých rozložení (Layouts) stránky za pomocí tématu Omega. Každé z těchto rozložení implementuje vlastní šablony (viz. dále) pro generování html kódu a tím umožňuje kompletně změnit výsledný vygenerovaný kód. Při vývoji je potřeba respektovat určitá pravidla, jako je počet a pojmenování segmentů, které musí odpovídat tématu vzhledu, ve kterém budou využita. Přepínání mezi rozloženími stránky se definují za pomocí modulu Context a jeho podmodulu Context Layouts (sekce \ref{subsec:context}). Zatímco pro všechny stránky portálu je tedy využito rozložení s názvem Blue (podle vzhledu stránek), v případě že se jedná o stránku implementující připojení ke vzdálené ploše (identifikace probíhá na základě \gls{url}), systém přepne na rozložení Guacamole (dle využité technologie).

Při vlastním vývoji vzhledu portálu je vhodné využívat technologií a rozhraní dostupných v \gls{cms} Drupal. Vývoj probíhá ve třech úrovních.

\begin{enumerate}
  \item Je potřeba generovat vhodný HTML kód. K tomu v Drupalu slouží systém přepisování šablonovacích souborů (přípona .tpl.php) pro jednotlivé prvky webu. V základu poskytuje Drupal stromovou strukturu, začínající u celého dokumentu (html.tpl.php), který obsahuje všechen obsah včetně hlaviček, a pokračující přes stránku (page.tpl.php) až po jednotlivé elementy (např. fieldset.tpl.php). Většina těchto šablon je definována přímo v jádru systému a všechny lze jednoduchým způsobem (na základě jména) přepsat. Zároveň je možné vytvářet úplně nové šablony. Šablonám lze zároveň i upravit data, která do nich budou vložena a to za pomoci implementace funkcí \_preprocess a \_process a jejich vložení do kódu tématu vzhledu. 
  
  \item Výstup html lze měnit přepsáním funkcí formátujících jednoduché elementy stránky (například seznam elementů ul). Typicky jsou tyto prvky generovány funkcí theme([jméno elementu], [proměnné]), která může být přepsána buď v modulu, nebo tématu vzhledu, jako [jméno modulu]\_[jméno elementu]() - například \texttt{function omega\_fieldset(\$variables)}. Modul Omega nad tímto rozhraním navíc poskytuje možnosti všechny tyto funkce vkládat do samostatných souborů pojmenovaných dle daných konvencí, čímž dále zpřehledňuje výsledný kód (výše zmíněný kód by byl umístěn v souboru \texttt{theme/container.theme.inc}).
  \item Na HTML kód se nanášejí \gls{css} styly. Jak je popsáno v sekci \ref{subsec:sass}, je výhodnější využívat nástroj SASS, který je do jazyka \gls{css} kompilován a poskytuje více možností. Vývoj probíhá dle standartů určených v tématu vzhledu Omega (viz. \ref{subsec:omega}) a je definován v rámci několika SASS souborů, kde do každého z nich je seskupena určitá množina nastavení.
\end{enumerate}

\begin{figure}[]
  \includegraphics[width=12cm]{img/drupal-theme-architecture-crop.pdf}
  \caption{Architektura vykreslení HTML kódu stránky}
  \label{fig:theme_architecture}
\end{figure}  

V repozitáři je již dostupná kompilovaná verze \gls{css} souborů, ale v případě nutnosti je možné je znovu vygenerovat. K tomu je nutno mít nainstalován programovací jazyk \emph{Ruby}, \emph{COMPASS} (sekce \ref{subsec:compass}) a \gls{bundler}.

\begin{enumerate}
  \item \texttt{\$ sudo gem install bundler}
  \item \texttt{\$ bundle install}
  \item \texttt{\$ compass watch}
\end{enumerate}

\emph{COMPASS} bude od této chvíle sledovat obsah složky a v případě jakýchkoliv změn automaticky aktualizovat výsledné \emph{CSS} soubory.

\section{Nasazení (deployment) a jeho možnosti}

Pro správu \gls{deployment} je využíván nástroj Phing (kapitola ..), umožňující definování pravidel v souboru build.xml. Kromě tohoto souboru je také dostupný soubor \linebreak \texttt{phing/\-default.build.properties} obsahující výchozí nastavení projektu. Při první inicializaci jakéhokoliv cíle (target) se zkontroluje přítomnost souboru \linebreak \texttt{build.properties} a v případě jeho absence se na jeho místo zkopíruje výše zmíněný soubor obsahující výchozí hodnoty. Pokud soubor není dostupný a není potřeba vytvářet instalaci, je vhodné spustit příkaz \texttt{phing void} který pouze připraví tento konfigurační soubor. 

V konfiguračním souboru je vhodné zadat nastavení prostředí, na kterém se bude portál instalovat (cesty k souborům, nastavení databáze a pod.) - více informací je v samotném souboru.

\begin{description}
  \item[void] \hfill \\
  Pouze provede inicializaci konfiguračních souborů a ukončí se.  
  \item[guacamole] 
  Vytvoří konfigurační soubory projektu Guacamole ve správných složkách (je zkopírován soubor \texttt{phing/default.guacamole.properties}.
  \item[deploy]
  Překopíruje aktuální moduly na správné místo a zajistí správné nastavení portálu dle aktuálního nastavení (převážně využití Features).
  \item[install]
  Provede kompletní instalaci portálu dle nastavení v konfiguračních souborech - překopíruje moduly a témata specifická pro toto řešení, stáhne potřebné zdrojové kódy Drupalu z internetu a nainstaluje Drupal, kterému následně nastaví správné téma vzhledu a povolí potřebné moduly.
\end{description}

\section{Výkonové optimalizace}
Ideálně si projít nějaké výzkumy Googlu ohledně odezvy a jejího dopadu na uživatele

\cite{high-performance-drupal}

Popsat agregaci JS, CSS a podobne 
\cite{website:drupal:optimizing}

\chapter{Implementace připojení ke vzdálené ploše}
\label{chap:implementace-guacamole}
Implementace připojení ke vzdálené ploše probíhala v několika fázích. Nejdříve bylo potřeba spustit řešení \emph{Guacamole} lokálně a ujistit se, že je použitelné pro naše potřeby. Bylo také nutné analyzovat, jaké všechny možnosti poskytuje a které vlastnosti jsou potřebné a které přebytečné. Jako přebytečné byly idnetifikovány vlastnosti poskytující správu uživatelů, připojení a jejich skupin. Dále je implementována podpora klávesnici na obrazovce, ukládání snímků obrazovky a mnoho dalších rozšířených možností, které nebyly do základní verze vyžadovány a byly spíše kontraproduktivní s ohledem na zvýšení komplexnosti kódu a debugování celého řešení. 

Jakmile byla funkčnost ověřena, bylo nutné celou klientskou část řešení přesunout do modulu v Drupalu a upravit pro tamější podmínky. Bylo nutné změnit HTML výstup, kdy Drupal si v základu buduje strukturu pomocí několika šablonových souborů pro stránku, tělo a pod. Bylo nutné přidat JavaScriptové knihovny z \emph{Guacamole}, odstranit nepotřebný kód a upravit vše k funkčnosti v prostředí s použitím \gls{cors}. 

Nástroj \emph{Guacamole} poskytuje mnohá nastavení a pro ulehčení byla do modulu přidána administrační stránka, která přímo zapisuje do konfiguračního souboru a tím eliminuje potřebu uživatelského připojení na server a manuální konfigurace. 

\section{Ostranění autentizace uživatelů a vytvoření vlastní autorizace}
\emph{Guacamole} v základu poskytuje API pro správu uživatelských účtů a jejich připojení. Jsou implementovány dvě metody, jedna spoléhá na uložení dat do konfiguračního souboru, zatímco druhá pracuje s MySQL databází. Obě rozšiřují třídu \emph{SimpleAuthenticationProvider}, ke které přidávají potřebnou funkcionalitu. 

Pro potřeby tohoto projektu však bylo potřeba vždy pouze jediné připojení, pro jednotlivé uživatele se měnilo pouze jejich přihlašovací jméno. Byla vytvořena nová třída \emph{DrupalAuthenticationProvider}, která načítá adresu serveru a port přímo z konfiguračního souboru pro všechna připojení a vrací je Guacamole k dalšímu zpracování. 

Pro nastavení přihlašovacích údajů je potřeba se nejdříve k serveru přihlásit. Pro minimalizaci úkonů potřebných ke každému připojení ke vzdálené ploše a také k zjednodušení celého procesu si musí každý uživatel portálu uložit své přihlašovací údaje do svého profilu, odkud jsou pak použity při přihlášení ke Guacamole Servletu. K tomu je použito dvou služeb - \emph{connect} a {login}. Proces je znázorněn na diagramu \ref{fig:login_process}. 

Komunikace je započata přímo v klientském prohlížeči z JavaScriptu. Po přístupu na stránku /aspi je nejdříve kontaktována pomocí technolie AJAX stránka /aspi/ajax na portálu, která se připojí ke servletu Login. Guacamole Servlet si vytvoří nové \gls{session} a jeho identifikátor odešle zpět PHP kódu. Pokud vše proběhne v pořádku, je identifikátor sezení odeslán pomocí set-cookie hlavičky zpět stránce /aspi, kde si informaci uloží prohlížeč jako cookie pro pozdější komunikaci. V případě jakékoliv chyby je chybové hlášení vypsáno na obrazovce a zalogováno v systému bez jakýchkoliv dalších pokusů o připojení, uživatel je tedy nucen buď stránku obnovit, nebo kontaktovat správce stránek. Pokud však vše proběhne v pořádku, je inicializována smyčka přímých volání služby \emph{tunnel} poskytované Guacamole Servlet kontejnerem za pomocí technologie \gls{cors}. Komunikace probíhá za pomocí Guacamole protokolu a ten je poté konvertován na grafický výstup na obrazovce uživatele, který je od této chvíle schopen pomocí myši a klávesnice ovládat vzdálenou plochu a tím i gls{aspi}.

\begin{figure}[]
  \includegraphics[width=12cm]{img/login-process-crop.pdf}
  \caption{Proces připojení ke vzdálené ploše}
  \label{fig:login_process}
\end{figure}  

\section{Přesun JavaScript kódu a CSS stylů do modulu Drupalu}
Pro ulehčení správy \gls{session}, které by jinak muselo být synchronizováno mezi dvěma stránkami, byl veškerý kód stránky z Guacamole JAVA projektu přesunut do modulu v drupalu. Do složky js/lib byly přesunuty všechny zdrojové kódy z knihovny Guacamole-js a byla vytvořena nová stránka /aspi poskytující stejnou funkcionalitu jako JAVA modul. Spolu s javascriptem bylo nutné přesunout i základní css styly pro zachování formátu vykreslování okna vzdálené plochy. Jak knihovny, tak styly jsou importovány pouze na stránce /aspi pro snížení náročnosti běhu celé platformy, která by jinak byla nucena načítat nepotřebné soubory. 

Drupal v základu vytváří komplexní html výstup včetně loga, základní struktury a podobně, kdežto guacamole vyžaduje pro vykreslení vzdálené plochy velmi jednoduchou strukturu obsahující prakticky jen plátno pro vykreslování výstupu a nastavení v hlavičce. Zatímco údaje do hlavičky lze přidat jednoduše pomocí implementace hook\_api drupalu (viz. sekce \ref{sec:technologies}), změnu výstupu bylo nutné implementovat pomocí kombinace několika technik. Téma vzhledu \emph{Omega} ve verzi 4.x poskytuje funkcionalitu rozdílných rozvržení (layouts) dostupných pro různé části stránky. Tato možnost lze skombinovat s modulem \emph{Context} a jeho podmodulem \emph{Context Omega}, který poskytuje přemostění mezi modulem a tématem vzhledu a tím i možnost změny rozvržení podle definovaných pravidel. V případě tohoto projektu stačilo nadefinovat pravidlo změny pro url /aspi na které se automaticky přepne rozvržení na minimalistické rozvržení nazvané \emph{Guacamole}. Toto nastavení je i exportováno do konfiguračního modulu esf\_feature.

\section{Implementace CORS}
Pro možnost komunikace mezi více doménami bylo potřeba implementovat \gls{cors} (viz. sekce \ref{sec:technologies}). Hlavní změna je u zpracování požadavků na straně serveru, k čemuž musela být do projektu přidána a nastavena knihovna k jeho zpracování. Pro JAVA EE byla využita knihovna CORS filter od [d]zhuvinov  [s]oftware\cite{website:cors-filter}, která tuto implementaci poskytuje ve formě knihovny. Na stránkách je i detailně popsáno nastavení a použití, kdy v připadě tohoto serveru bylo nutno omezit zdroj \gls{cors} požadavků na jedinou doménu a více nebylo nutno se o bezpečnost obávat.

Druhá část implementace je umožnění cross-domain požadavků v JavaScriptu. Knihovna Guacamole-js musela být lehce přepsána v místech, kde byla použita volání XMLHttpRequest, která jsou v tomto případě volána asynchronně. Použité řešení bohužel nelze aplikovat na \gls{ie}, neboť XDomainRequest nepodporuje asynchronní zpracování a bude nutné toto řešení ještě přepracovat do funkční podoby.

\section{Implementace úpravy nastavení z administrace portálu}
Guacamole typicky poskytuje nastavení pomocí konfiguračního souboru umístěného buď ve složce instalace, nebo v domácí složce uživatele, který program spouští. Jelikož je ke změně těchto konfiguračních souborů nutné přistupovat přímo k serveru, rozhodl jsem se tuto administrační část přepracovat a zpřístupnit přímo z administrace portálu. Ta nabízí dvě základní stránky. Na stránce Nastavení (/config/esf/settings) je možné upravit základní nastavení projektu jako je URL adresa pro připojení ke Guacamole Servlet službám či právě přístup ke konfiguračnímu souboru. Ten se typicky jmenuje guacamole.properties a nachází se ve složce /srv/guacamole/. Je nutné se ujistit, že jak Guacamole Servlet (TomCat) tak \gls{esf} Portál (Apache) mají k souboru přístup a mohou jej upravovat.

Na stránce nastavení Připojení ke vzdálené ploše (\texttt{/config/esf/remote}) lze za předpokladu, že je vše správně nastaveno, upravovat nastavení Guacamole samotného a url pro připojení k serveru vzdálené plochy - což je v našem případě ASPI. Dále je nutné nastavit port pro připojení ke Guacd démonovi poskytujícímu proxy pro připojení ke vzdálené ploše, uživatelské jméno a heslo je však převzato z nastavení každého jednotlivého uživatele.

\section{Vývoj tématu vzhledu}
Implementace vzhledu probíhá na dvou úrovních. 
\begin{enumerate}
  \item Je potřeba generovat vhodný HTML kód. K tomu v Drupalu slouží systém přepisování šablonovacích souborů (přípona .tpl.php) pro jednotlivé prvky webu. V základu poskytuje Drupal stromovou strukturu, začínající u celého dokumentu (html.tpl.php), který obsahuje všechen obsah včetně hlaviček, a pokračující přes stránku (page.tpl.php) až po jednotlivé elementy (např. fieldset.tpl.php). Většina těchto šablon je definována přímo v jádru systému a všechny lze jednoduchým způsobem (na základě jména) přepsat. Zároveň je možné vytvářet úplně nové šablony. Šablonám lze zároveň i upravit data, která do nich budou vložena a to za pomoci implementace funkcí \_preprocess a \_process a jejich vložení do kódu tématu vzhledu. 
  Další způsob, jak změnit výstup html je přepsání funkcí formátujících jednoduché elementy stránky (například seznam elementů ul). Typicky jsou tyto prvky generovány funkcí theme([jméno elementu], [proměnné]), která může být přepsána buď v modulu, nebo tématu vzhledu, jako [jméno modulu]\_[jméno elementu]() - například \texttt{function omega\_fieldset(\$variables)}. Modul Omega nad tímto rozhraním navíc poskytuje možnosti všechny tyto funkce vkládat do samostatných souborů pojmenovaných dle daných konvencí, čímž dále zpřehledňuje výsledný kód (výše zmíněný kód by byl umístěn v souboru \texttt{theme/container.theme.inc}).
  \item Na HTML kód se nanášejí \gls{css} styly. Jak je popsáno v sekci \ref{subsec:sass}, je výhodnější využívat nástroj SASS, který je do jazyka \gls{css} kompilován a poskytuje více možností. Vývoj probíhá dle standartů určených v tématu vzhledu Omega (viz. \ref{subsec:omega}) a je definován v rámci několika SASS souborů, kde do každého z nich je seskupena určitá množina nastavení.
\end{enumerate}

\begin{figure}[]
  \includegraphics[scale=0.85]{img/drupal-theme-architecture-crop.pdf}
  \caption{Architektura vykreslení HTML kódu stránky}
  \label{fig:theme_architecture}
\end{figure}  

V repozitáři je již dostupná kompilovaná verze \gls{css} souborů, ale v případě nutnosti je možné je znovu vygenerovat. K tomu je nutno mít nainstalován programovací jazyk Ruby a nástroje COMPASS (sekce \ref{subsec:compass}) a \gls{bundler}.

\begin{enumerate}
  \item \texttt{\$ sudo gem install bundler}
  \item \texttt{\$ bundle install}
  \item \texttt{\$ compass watch}
\end{enumerate}

COMPASS bude od této chvíle sledovat obsah složky a v případě jakýchkoliv změn automaticky aktualizovat výsledné CSS soubory.
