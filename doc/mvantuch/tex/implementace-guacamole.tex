\chapter{Implementace připojení ke vzdálené ploše}
\label{chap:implementace-guacamole}
Implementace připojení ke vzdálené ploše probíhala v několika fázích. Nejdříve bylo potřeba spustit řešení \emph{Guacamole} lokálně a ujistit se, že je použitelné pro naše potřeby. Bylo také nutné analyzovat, jaké všechny možnosti poskytuje a které vlastnosti jsou potřebné a které přebytečné. Jako přebytečné byly idnetifikovány vlastnosti poskytující správu uživatelů, připojení a jejich skupin. Dále je implementována podpora klávesnici na obrazovce, ukládání snímků obrazovky a mnoho dalších rozšířených možností, které nebyly do základní verze vyžadovány a byly spíše kontraproduktivní s ohledem na zvýšení komplexnosti kódu a debugování celého řešení. 

Jakmile byla funkčnost ověřena, bylo nutné celou klientskou část řešení přesunout do modulu v Drupalu a upravit pro tamější podmínky. Bylo nutné změnit HTML výstup, kdy Drupal si v základu buduje strukturu pomocí několika šablonových souborů pro stránku, tělo a pod. Bylo nutné přidat JavaScriptové knihovny z \emph{Guacamole}, odstranit nepotřebný kód a upravit vše k funkčnosti v prostředí s použitím \gls{cors}. 

Nástroj \emph{Guacamole} poskytuje mnohá nastavení a pro ulehčení byla do modulu přidána administrační stránka, která přímo zapisuje do konfiguračního souboru a tím eliminuje potřebu uživatelského připojení na server a manuální konfigurace. 


\section{Architektura připojení ke vzdálené ploše}
Základně poskytuje Guacamole komplexní řešení na propojení ke vzdálené ploše včetně stránky pro management uživatelů, kde každý z nich si může pro sebe vytvořit své vlastní připojení, u každého z nich se zobrazuje náhled a podobně. Jak je zobrazeno na diagramu \ref{fig:arch_guacamole}, součástí Guacamole je i webová aplikace v jazyce HTML, která implementuje knihovnu Guacamole-js a tím i komunikaci s JAVA Servletem. Pro potřeby tohoto projektu však bylo nutné tyto JavaScriptové knihovny převzít a implementovat do modulu pro Drupal. 

\begin{figure}[htp] 
  \centering{\includegraphics[width=12cm]{img/architecture-guacamole-crop.pdf}}
  \caption{Neupravená architektura a rozčlenění řešení}
  \label{fig:arch_guacamole}
\end{figure}  

Díky skutečnosti, že již nutně neběží JAVA Server a JAVA EE Web aplikace na stejném serveru, či url, bylo nutné změnit architekturu AJAXových požadavků. Ty totiž musejí v typickém případě být zasílány pouze na stejnou doménu kvůli prevenci \gls{xss}. Pro vyřešení tohoto problému byla změněna implementace na CORS požadavky, k čemuž musel být uzpůsoben i JAVA Servlet kontejner, u kterého musely být povoleny požadavky z dané domény. 

Uživatel proto pracuje základně pouze s portálem postaveném na Drupalu, do kterého byl přidán modul implementující stránku určenou ke komunikaci s Guacamole Servlet kontejnerem za pomoci upravené JavaScriptové knihovny Guacamole-js, jak je vyobrazeno na diagramu \ref{fig:arch_drupal}.

\begin{figure}[htp] 
  \centering{\includegraphics[width=12cm]{img/architecture-drupal-crop.pdf}}
  \caption{Architektura přizpůsobená pro portál ESF}
  \label{fig:arch_drupal}
\end{figure}  

\section{Ostranění autentizace uživatelů a vytvoření vlastní autorizace}
\emph{Guacamole} v základu poskytuje API pro správu uživatelských účtů a jejich připojení. Jsou implementovány dvě metody, jedna spoléhá na uložení dat do konfiguračního souboru, zatímco druhá pracuje s MySQL databází. Obě rozšiřují třídu \emph{SimpleAuthenticationProvider}, ke které přidávají potřebnou funkcionalitu. 

Pro potřeby tohoto projektu však bylo potřeba vždy pouze jediné připojení, pro jednotlivé uživatele se měnilo pouze jejich přihlašovací jméno. Byla vytvořena nová třída \emph{DrupalAuthenticationProvider}, která načítá adresu serveru a port přímo z konfiguračního souboru pro všechna připojení a vrací je Guacamole k dalšímu zpracování. 

Pro nastavení přihlašovacích údajů je potřeba se nejdříve k serveru přihlásit. Pro minimalizaci úkonů potřebných ke každému připojení ke vzdálené ploše a také k zjednodušení celého procesu si musí každý uživatel portálu uložit své přihlašovací údaje do svého profilu, odkud jsou pak použity při přihlášení ke Guacamole Servletu. K tomu je použito dvou služeb - \emph{connect} a {login}. Proces je znázorněn na diagramu \ref{fig:login_process}. 

Komunikace je započata přímo v klientském prohlížeči z JavaScriptu. Po přístupu na stránku /aspi je nejdříve kontaktována pomocí technolie AJAX stránka /aspi/ajax na portálu, která se připojí ke servletu Login. Guacamole Servlet si vytvoří nové \gls{session} a jeho identifikátor odešle zpět PHP kódu. Pokud vše proběhne v pořádku, je identifikátor sezení odeslán pomocí set-cookie hlavičky zpět stránce /aspi, kde si informaci uloží prohlížeč jako cookie pro pozdější komunikaci. V případě jakékoliv chyby je chybové hlášení vypsáno na obrazovce a zalogováno v systému bez jakýchkoliv dalších pokusů o připojení, uživatel je tedy nucen buď stránku obnovit, nebo kontaktovat správce stránek. Pokud však vše proběhne v pořádku, je inicializována smyčka přímých volání služby \emph{tunnel} poskytované Guacamole Servlet kontejnerem za pomocí technologie \gls{cors}. Komunikace probíhá za pomocí Guacamole protokolu a ten je poté konvertován na grafický výstup na obrazovce uživatele, který je od této chvíle schopen pomocí myši a klávesnice ovládat vzdálenou plochu a tím i gls{aspi}.

\begin{figure}[]
  \includegraphics[width=12cm]{img/login-process-crop.pdf}
  \caption{Proces připojení ke vzdálené ploše}
  \label{fig:login_process}
\end{figure}  

\section{Přesun JavaScript kódu a CSS stylů do modulu Drupalu}
Pro ulehčení správy \gls{session}, které by jinak muselo být synchronizováno mezi dvěma stránkami, byl veškerý kód stránky z Guacamole JAVA projektu přesunut do modulu v drupalu. Do složky js/lib byly přesunuty všechny zdrojové kódy z knihovny Guacamole-js a byla vytvořena nová stránka /aspi poskytující stejnou funkcionalitu jako JAVA modul. Spolu s javascriptem bylo nutné přesunout i základní css styly pro zachování formátu vykreslování okna vzdálené plochy. Jak knihovny, tak styly jsou importovány pouze na stránce /aspi pro snížení náročnosti běhu celé platformy, která by jinak byla nucena načítat nepotřebné soubory. 

Drupal v základu vytváří komplexní html výstup včetně loga, základní struktury a podobně, kdežto guacamole vyžaduje pro vykreslení vzdálené plochy velmi jednoduchou strukturu obsahující prakticky jen plátno pro vykreslování výstupu a nastavení v hlavičce. Zatímco údaje do hlavičky lze přidat jednoduše pomocí implementace hook\_api drupalu (viz. sekce \ref{sec:technologies}), změnu výstupu bylo nutné implementovat pomocí kombinace několika technik. Téma vzhledu \emph{Omega} ve verzi 4.x poskytuje funkcionalitu rozdílných rozvržení (layouts) dostupných pro různé části stránky. Tato možnost lze skombinovat s modulem \emph{Context} a jeho podmodulem \emph{Context Omega}, který poskytuje přemostění mezi modulem a tématem vzhledu a tím i možnost změny rozvržení podle definovaných pravidel. V případě tohoto projektu stačilo nadefinovat pravidlo změny pro url /aspi na které se automaticky přepne rozvržení na minimalistické rozvržení nazvané \emph{Guacamole}. Toto nastavení je i exportováno do konfiguračního modulu esf\_feature.

\section{Implementace CORS}
Pro možnost komunikace mezi více doménami bylo potřeba implementovat \gls{cors} (viz. sekce \ref{sec:technologies}). Hlavní změna je u zpracování požadavků na straně serveru, k čemuž musela být do projektu přidána a nastavena knihovna k jeho zpracování. Pro JAVA EE byla využita knihovna CORS filter od [d]zhuvinov  [s]oftware\cite{website:cors-filter}, která tuto implementaci poskytuje ve formě knihovny. Na stránkách je i detailně popsáno nastavení a použití, kdy v připadě tohoto serveru bylo nutno omezit zdroj \gls{cors} požadavků na jedinou doménu a více nebylo nutno se o bezpečnost obávat.

Druhá část implementace je umožnění cross-domain požadavků v JavaScriptu. Knihovna Guacamole-js musela být lehce přepsána v místech, kde byla použita volání XMLHttpRequest, která jsou v tomto případě volána asynchronně. Použité řešení bohužel nelze aplikovat na \gls{ie}, neboť XDomainRequest nepodporuje asynchronní zpracování a bude nutné toto řešení ještě přepracovat do funkční podoby.

\section{Implementace úpravy nastavení z administrace portálu}
Guacamole typicky poskytuje nastavení pomocí konfiguračního souboru umístěného buď ve složce instalace, nebo v domácí složce uživatele, který program spouští. Jelikož je ke změně těchto konfiguračních souborů nutné přistupovat přímo k serveru, rozhodl jsem se tuto administrační část přepracovat a zpřístupnit přímo z administrace portálu. Ta nabízí dvě základní stránky. Na stránce Nastavení (/config/esf/settings) je možné upravit základní nastavení projektu jako je URL adresa pro připojení ke Guacamole Servlet službám či právě přístup ke konfiguračnímu souboru. Ten se typicky jmenuje guacamole.properties a nachází se ve složce /srv/guacamole/. Je nutné se ujistit, že jak Guacamole Servlet (TomCat) tak \gls{esf} Portál (Apache) mají k souboru přístup a mohou jej upravovat.

Na stránce nastavení Připojení ke vzdálené ploše (\texttt{/config/esf/remote}) lze za předpokladu, že je vše správně nastaveno, upravovat nastavení Guacamole samotného a url pro připojení k serveru vzdálené plochy - což je v našem případě ASPI. Dále je nutné nastavit port pro připojení ke Guacd démonovi poskytujícímu proxy pro připojení ke vzdálené ploše, uživatelské jméno a heslo je však převzato z nastavení každého jednotlivého uživatele.
