\chapter{Implementace portálu v Drupalu}
\label{chap:implementace-drupal}

Pro převedení stávajícího řešení na novou verzi \emph{Drupalu} bylo nutné vykonat několik kroků. Nejprve byla data z produkčního prostředí přesunuta na testovací server, kde bylo řešení zanalyzováno co se týče implementační stránky (k běžícímu portálu nebylo možné přistupovat s administrátorským účtem). Jakmile byly stránky zanalyzovány, byl započat proces jejich aktualizace na novou verzi Drupalu při zachování veškerého obsahu. Tento postup se ukázal jako kontraproduktivní, neboť i když byl obsah stránek převeden, bylo potřeba kompletně změnit jeho strukturu spolu s konverzí na jiné typy obsahu s jinými cílovými polemi. Všechna data tak nakonec byla přesunuta ručně.

\section{Aktualizace Drupal 6 na Drupal 7}
\label{sec:aktualizace}
Sedmá verze Drupalu komplexně přepracovala celý kód redakčního systému a zohlednila v té době relevantní požadavky na moderní \gls{cms}. Mnoho modulů bylo přesunuto přímo do jádra systému - například podpora vytváření vlastních polí obsahu, pro což byl v \emph{Drupalu} 6 potřeba modul \emph{CCK} (Content Creation Kit). Množství změn je jednoduše vidět i ve velikosti zdrojových souborů, kdy verze 7 obsahuje 12.8~MB zdrojového kódu oproti 3,7~MB u verze 6.

Kromě funkcionality přidané do jádra, které by bylo možné dosáhnout i za pomocí modulů a předchozí verze, byly přepracovány i mnohé části funkcionality jádra systému~\cite{website:drupal-comparison}. Pro toto řešení jsou relevantní následující vylepšení:

\begin{itemize}
  \item podpora implementace vyrovnávací paměti za pomocí reverzní proxy (např. Varnish)
  \item Entity API - unifikace elementů stránek a poskytnutí \gls{api} ulehčujícího vytváření nových typů obsahu a tříd pro ulehčení práce~\cite{drupal-entities}
  \item možnost přidávat pole ke všem entitám (uživatelé, slovníky, \dots)
  \item podpora různých databází se stejným zdrojovým kódem v \emph{PHP} (abstrakce nad dotazy do databáze a využití knihovny \emph{PDO})
  \item zlepšená podpora \emph{AJAX} volání
  \item celkově propracovanější kód (vhodný pro vývoj dalších modulů)
\end{itemize}

V době práce na informačním systému již byl dostupný i \emph{Drupal} verze 8, který však nebyl stabilní. Některé jeho vlastnosti byly zpětně implementovány jako moduly pro \emph{Drupal} 7 a jakožto implementace moderních technologií byly zahrnuty i v tomto projektu (například NavBar\footnote{https://drupal.org/project/navbar} poskytující responsivní administrační panel, filtrace modulů, či administrační téma Ember\footnote{https://drupal.org/project/ember} postavené na SASS). 

\begin{table}
  \caption{Porovnání verzí modulů mezi Drupalem 6 a 7}
  \begin{tabular}{ | p{5cm} | p{2.5cm} | p{2.5cm} | c | }
    \hline 
    Jméno modulu & Drupal 6 & Drupal 7 & stav  \\ \hline 
    Backup and Migrate & 2.7 & 2.7 & \checkmark \\ \hline
    Colorbox & 1.6 & 2.4 & \checkmark \\ \hline
    CCK & 2.9 & jádro & \checkmark \\ \hline
    Custom Breadcumbs & 1.5 & 1.x-alpha3 & \\ \hline
    DB Maintenance & 1.4 & 1.1 & \checkmark \\ \hline
    DHTML Menu & 3.5 & 1.0-beta1 & \\ \hline 
    Email Field & 1.4 & 1.2 & \checkmark \\ \hline
    File (Field Paths & 1.5 & 1.0-beta4 & \\ \hline
    FileField & 3.11 & jádro & \checkmark \\ \hline
    Front Page & 1.3 & 2.4 & \checkmark \\ \hline
    ImageAPI & 3.11 & jádro & \checkmark \\ \hline
    ImageCache & 2.0-rc1 & jádro & \checkmark \\ \hline
    IMCE & 2.5 & 1.7 & \checkmark \\ \hline
    IMCE Wysiwyg bridge & 1.1 & 1.0 & \checkmark \\ \hline
    jCarousel & 2.6 & 2.6 & \checkmark \\ \hline
    Link & 2.10 & 1.1 & \checkmark \\ \hline
    Localization Update & 2.10 & 1.1 & \checkmark \\ \hline
    Menu Attributes & 2.0-beta1 & 1.0-rc2 & \checkmark \\ \hline
    Menu Block & 2.4 & 2.3 & \checkmark \\ \hline
    Pathauto & 1.6 & 1.2 & \checkmark \\ \hline
    Taxonomy Breadcrumb & 1.1 & - & \\ \hline
    Token & 1.19 & 1.5 & \checkmark \\ \hline
    Transliteration & 3.1 & 3.1 & \checkmark \\ \hline
    Views & 2.16 & 3.7 & \\ \hline
    Views Search & 1.0 & - & \\ \hline
    WYSIWYG & 2.4 & 2.2 & \checkmark \\ \hline
  \end{tabular}
\end{table}

\section{Odebrané moduly}
Funkcionalita některých z modulů byla v sedmé verzi přesunuta přímo do jádra a proto nebylo jejich využití již potřebné - byly využity jejich ekvivalenty z jádra. Kromě těchto byly odebrány i další moduly, kdy některé nejsou pro novou verzi vůbec dostupné. Dále byly odstraněny moduly, jejichž využití již postrádalo smysl - z důvodu změny a zjednodušení struktury menu předchozí implementace. Několik dalších modulů bylo odstraněno, neboť nebyly na stránkách vůbec využity a zpomalovaly běh portálu. 

\subsubsection*{\textbf{Views Search} \hfill \emph{https://drupal.org/project/views\_search}}
Rozšiřuje funkcionalitu modulu \emph{Views} a jejich obsahových filtrů, pro novou verzi již není modul dostupný a funkcionalita není ani vyžadována.

\subsubsection*{\textbf{Front Page} \hfill \emph{https://drupal.org/project/front}}
Slouží k nastavení výchozí stránky pro různé uživatelské role. Z důvodu plánování kompletně nové implementace domovské stránky byla tato funkcionalita prozatím odstraněna s možností opětovného zapnutí v budoucnosti.

\subsubsection*{\textbf{Menu Attributes} \hfill \emph{https://drupal.org/project/menu\_attributes}}
Umožňuje přiřazení attributů k položkám menu. Ty pak mohou být použity buď pro stylování, nebo navázány na JavaScript. Tato funkcionalita opět nebyla vyžadována.

\subsubsection*{\textbf{Taxonomy Breadcrumb} \hfill \emph{https://drupal.org/project/taxonomy\_breadcrumb}}
Generuje drobečkovou navigaci za pomoci slovníků definovaných v systému a jejich hierarchie. Hierarchie tím pádem může být velmi jendoduše upravena - toto však již není po úpravě systému potřeba s ohledem na zjednodušenou strukturu.

\subsubsection*{\textbf{Custom Breadcrumbs} \hfill \emph{https://drupal.org/project/custom\_breadcrumbs}}
Umožňuje detailní nastavení drobečkové navigace, která již není potřebná.

\subsubsection*{\textbf{DHTML Menu} \hfill \emph{https://drupal.org/project/dhtml\_menu}}
Mění funkcionalitu menu - při prvním kliknutí se namísto přesunutí na danou stránku pouze rozbalí menu s potomky daného odkazu a až při druhém kliknutí se přejde na danou stránku - tato funkcionalita zřejmě nebyla využívána.

\subsubsection*{\textbf{jCarousel} \hfill \emph{https://drupal.org/project/jcarousel}}
Slouží k propojení se stejnojmennou JavaScriptovou knihovnou poskytující zobrazení obrázků v ovladatelném ,,kolotoči''. S ohledem na aktuální absenci jakékoliv grafiky byl modul shledán nepotřebným.
 
\subsubsection*{\textbf{Colorbox} \hfill \emph{https://drupal.org/project/colorbox}}
Podobně jako jCarousel slouží k propojení s knihovnou určenou ke zobrazování obrázků a ani tento modul nebyl aktivně využíván.

\section{Nově přidané moduly}
\label{sec:nove-moduly}
Do projektu byly zahrnuty další moduly, některé z nich rozšiřující funkcionalitu řešení, jiné za účelem zjednodušení vývoje a údržby. Další byly přidány kvůli kompatibilitě s propojením se vzdálenou plochou za pomocí nástroje \emph{Guacamole}.

\subsubsection*{\textbf{Context} \hfill \emph{https://drupal.org/project/context}} 
\label{subsec:context}
Modul Context umožňuje definovat reakce v závislosti na splněných podmínkách prostředí portálu. Pro každou podmínku, kterou může být například role přihlášeného uživatele, aktuální stránka či jazyk, lze nadefinovat změny v portálu. Mezi dostupné akce patří nastavení drobečkové navigace, změna obsahu bloku a podobně. Podmínky lze mezi sebou vzájemně kombinovat a uzavírat do logických výrazů, stejně jako lze pro jeden logický výraz provézt více operací. Pro potřeby tohoto projektu je využíván hlavně podmodul \emph{Context Layouts} a modul \emph{Context Omega}\footnote{https://drupal.org/project/conext\_omega}, které dohromady umožňují změnu rozložení stránek.

\subsubsection*{\textbf{Kerberos Authentication} \hfill \emph{https://drupal.org/project/kerberos\_authentication}} 
Modul využívá protokol Kerberos k autentizaci uživatelů - v tomto případě za pomoci serveru Masarykovy Univerzity. Uživatelům, kteří ke připojení použijí své \gls{uco}, je automaticky vytvořen účet se stejným uživatelským jménem a mohou si poté změnit heslo speicficky pro tento portál. Modul dále povoluje vypnutí autentizace pomocí účtů pouze v portálu, tato vlastnost však není pro tento projekt potřebná.

\subsubsection*{\textbf{Features} \hfill \emph{https://drupal.org/project/features}} 
\label{subsec:features}
Jak je popsáno v kapitole~\ref{chap:vyvoj}, modul Features slouží k ulehčení vývoje a správy nastavení stránek. Základní myšlenkou je třívrstvé rozvržení systému, v první úrovni je jádro a jeho moduly, ve druhé rozšiřující moduly (tzv. Contrib) a ve třetí jsou rysy (Features), které v sobě zapouzdřují finální nastavení stránek. Ačkoliv se jedná také o moduly, jsou generovány systémem na stránce \texttt{admin/structure/features}. Zde je možné vytvořit nový rys a spravovat ty stávající. Ty mohou být v několika stavech v závislosti na aktuálním nastavení systému. Pokud bylo v sytému něco změněno od chvíle, kdy byl rys vygenerován a povolen (případně pouze povolen, pokud byl přenesen z jiného serveru), je možné dané změny buď vrátit ke stavu definovanému v rysu, nebo rys aktualizovat na novou verzi.

\subsubsection*{\textbf{Features Extra}\hfill \emph{https://drupal.org/project/features\_extra}} 
Jelikož modul \emph{Features} v sobě implementuje pouze základní funkcionalitu exportu typů obsahu a poskytuje \gls{api} pro další moduly, některé prvky webu nejsou exportovatelné. Tento modul možnosti exportu rozšiřuje o další možnosti, ze kterých je pro tento projekt relevantní export nastavení umístění bloků.

\subsubsection*{\textbf{@font-your-face}\hfill \emph{https://drupal.org/project/fontyourface}} 
Pro využití řezů písem třetích stran je možné postupovat několika způsoby - buď nahrát soubor obsahující písmo přímo na server, nebo se odkazovat na server třetí strany, který daný font poskytuje (např. Google Fonts API). Modul @font-your-face unifikuje tyto přístupy a poskytuje jednoduché rozhraní pro výběr typů a řezů písma a jejich použití na stránkách. Buď lze vše nastavit v administračním rozhraní, nebo lze poskytnutý kód jednoduše vložit do souboru definujícího téma vzhledu - například řádek \texttt{fonts[google\_fonts\_api][] = "BenchNine\&subset=latin-ext\#300"} automaticky načte z Google Fonts API font se jménem BenchNine v řezu 300.

\subsubsection*{\textbf{Entity API}\hfill \emph{https://drupal.org/project/entity}}
Jak je popsáno v sekci \ref{sec:aktualizace}, v předchozí verzi \emph{Drupalu} bylo možné vytvářet pouze uzly (node), které s sebou nesou nezanedbatelnou složitost v podobě vlastní \gls{url}, revizí a dalších informací. Ve verzi 7 je již možné vytvořit i čistou entitu, u které lze nastavit všechny vlastnosti, například zda ukládá informace o autorovi a podobně. V základu je tato implementace poněkud komplikovaná a není nijak lehce dostupná administrátorům, ani vývojářům. Modul \emph{Entity API} poskytuje unifikované rozhraní, umožňující jednoduchou definici entity na bázi zdrojového kódu\cite{drupal-entities}. 

\subsubsection*{\textbf{Entity Creation Kit} (ECK)\hfill \emph{https://drupal.org/project/eck}} 
Zatímco modul \emph{Entity API} poskytuje možnost definovat entity ve zdrojovém kódu, za pomoci modulu \emph{ECK} je možno je definovat z administračního rozhraní, čímž je celý proces tvorby značně zjednodušen. Definované entity lze exportovat za pomocí modulu \emph{Features} do kódu modulu a tím zabezpečit jejich verzování.

\subsubsection*{\textbf{Entity Reference}\hfill \emph{https://drupal.org/project/entity\_reference}}
Spolu s modulem \emph{Inline Entity Form}\footnote{https://drupal.org/project/inline\_entity\_form} poskytuje modul \emph{Entity Reference} uživatelsky přívětivou možnost ke vzájemnému propojení entit. Každá entita (takže i všechny uzly) může odkazovat na jednu či více entit jiného, či stejného typu. Propojené entity mohou být vytvářeny přímo na stránce, na které je upravována entita původní.

\subsubsection*{\textbf{Media} \hfill \emph{https://drupal.org/project/media}}
\label{subsec:media}
\gls{cms} systém \emph{Drupal} v základu poskytuje pouze jednoduchou funkcionalitu práce s obrázky. Zvuk, video ani dokumenty nejsou v aktuální verzi (7.x) podporovány vůbec. Zároveň uživatelé nemají mnoho možností operovat se soubory, které nahráli na server a tím pádem jsou často nuceni obsah nahrávat opakovaně, čímž docházi k jeho duplikaci. Tyto a mnohé další problémy se snaží řešit projekt \emph{Media}, poskytující uživatelům srozumitelné rozhraní a unifikaci práce s různými typy multimediálního obsahu. Pro svůj běh vyžaduje modul \emph{File Entity}\footnote{https://drupal.org/project/file\_entity}, který poskytuje možnosti práce se soubory jako entitami v systému.

\subsubsection*{\textbf{Views Bulk Operations} (VBO) \hfill \emph{https://drupal.org/project/views\_bulk\_operations}}
\label{subsec:vbo}
Modul \emph{Views}\footnote{https://drupal.org/project/views} rozšiřuje funkcionalitu o uživatelsky definovatelné pohledy na data (podobně jako views v databázi) a v dnešní době je nainstalován téměř na každé instalaci \emph{Drupalu} (ve verzi 8 je již obsažen v jádru). Tento modul však není sám schopen poskytovat operace nad více řádky najednou, s čímž pomáhá modul \emph{Views Bulk Operations} - ke každému řádku je přiřazeno zaškrtávací tlačítko a nad formulářem je zobrazen seznam operací, které lze nad elementy provádět.

\subsubsection*{\textbf{Workbench} \hfill \emph{https://drupal.org/project/workbench}}
\label{subsec:workbench}
Administrace \emph{Drupalu} je mnohdy kritizována pro její nízkou uživatelskou přívětivost a složitost, čemuž se snaží ulehčit tento modul. Přihlášenému uživateli s právem administrace či správy obsahu je umožněno přistoupit k ,,pracovnímu stolu'', stránkách postavených na modulu \emph{Views} a poskytujících přehled obsahu stránek - ať už se jedná o jakýkoliv nový obsah, nebo o obsah naposledy upravený daným uživatelem. Tato funkcionalita je dále rozšířena, jak je popsáno v kapitole \ref{chap:vyvoj}.

\subsubsection*{\textbf{Workbench Media} \hfill \emph{https://drupal.org/project/workbench\_media}}
Zatímco modul \emph{Workbench} poskytuje pohled na obsah stránek, \emph{Workbench Media}, jak již název nápovídá, přidává propojení s modulem \emph{Media} a tím pádem pohled na mediální obsah daného uživatele a celého portálu. 

\section{Struktura typů obsahu}
\label{subsec:typy-obsahu}

Kdybychom si představili typy obsahu ve světě objektově orientovaného programování, všechny by dědily vlastnosti ze základní třídy \emph{Entita}. Těmito vlastnostmi jsou typ, ID a několik dalších, relevantních pouze pro interní fungování redakčního systému. Z \emph{Entity} dědí třída \emph{Uzel}, která ji rozšiřuje o nadpis, adresu a stav. Pro lepší představu jsou vztahy popsány na diagramu \ref{fig:class-diagram}. 

\subsection*{Typy entit}

\begin{description}
  \item[Uzel] Stránka mající vlastní \gls{url} a obsah.
  \item[Položka] Entita s nadpisem použitelná k dalšímu rozšíření.
  \item[Osoba] Entita použitelná pro osobu, obsahující pole pro vložení jména.
  \item[Médium] Multimediální obsah - obraz, zvuk či video uložené na serveru, nebo referencované ze služby třetí strany.
  \item[Slovník] Záznam ve slovníku (česky také označován jako čísleník), který obsahuje definovanou množinu záznamů, které se typicky přiřazují jako značky dalším typům obsahu v rámci kategorizace.
\end{description}

\begin{figure}[htp] \centering{
\includegraphics[width=12cm]{img/class-diagram-crop.pdf}}
\caption{Diagram tříd popisující vztahy mezi typy obsahu}
\label{fig:class-diagram}
\end{figure}  

\subsection*{Typy obsahu}

\begin{description}
  \item[Novinka \emph{(uzel)}] Na stránkách jsou publikovány novinky (aktuality). Z typických témat lze zmínit \gls{esf}, portál samotný či jakékoliv dalších relevantní události. Tyto novinky jsou zobrazeny ve vysouvacím postranním panelu.
  
  \item[Předmět \emph{(uzel)}] Popisuje předmět vyučovaný v rámci fakulty \gls{esf} a tím vyjadřuje základní rozdělení celého portálu a slučuje hlavní část informací, o které se návštěvníci zajímají. Na stránce detailu předmětu je zobrazen ktátký popis, za kterým následuje tabulka se seznamem studijních textů a k nim se vztahujících materiálů. Materiály, které se vztahují přímo k předmětu jsou zobrazeny samostatně a celá stránka je uzavřana seznamem oborů které s daným předmětem souvisejí.

  \item[Obor práva \emph{(uzel)}] Zatímco některé předměty mohou souviset pouze s jedním oborem práva, jiné se mohou dotýkat téměř všech. Proto je důležité tento vztah vymodelovat a uživatelům poskytnout možnost se v něm jednoduše orientovat a přecházet mezi obsahem, který je právě zajímá. Na stránce oboru práva je zobrazen popis oboru a tabulka se seznamem předmětů, které se k němu vztahují.

  \item[Studijní text \emph{(uzel)}] Studijní text obsahuje informace vztahující se k jednomu celku výuky, typicky k jedné přednášce. Lze definovat zkrácený cíl přednášky, přidat k ní materiály a určit její pořadové číslo. U všech materiálů je vhodné zadat název pro správné zobrazení v tabulce detailu předmětu, na kterém se zobrazuje seznam těchto entit.

  \item[Příběh \emph{(uzel)}] Pro odreagování studentů je v jednom ze segmentů stránek zobrazován náhodný příběh z právního prostředí. 

  \item[Vtip \emph{(položka)}] V postranním panelu lze zobrazit vždy aktuální vtip z právního prostředí, který, podobně jako příběh, slouží k odreagování studentů. 

  \item[Vyučující \emph{(osoba)}] Entita popisující vyučujícího působícího na fakultě. Ke každému vyučujícímu je možné doplnit akademické tituly, které jsou při výstupu správně formátovány.
\end{description}

\section{Slovníky}
\label{sec:slovniky}

Pro kategorizaci a pro výběr možností bylo v požadavcích identifikováno několik slovníků. Díky vztahu k entitě (ze které záznamy slovníku dědí) může také pro své hodnoty definovat pole a tím rozšířit jejich funkcionalitu a množství informací.

\subsection*{Stupeň obecnosti}
Pro řazení oborů je potřeba uložit jejich obecnost, která je vyjádřena v několika stupních, které mohou být v případě potřeby přidávány. Každé entitě může být přiřazen jeden stupeň obecnosti.

\begin{list}{-}{Pole}
  \item Váha (váha) \hfill \\
    číselná hodnota použitelná k řazení
\end{list}

\begin{list}{-}{Hodnoty}
  \item Velmi obecný \emph{(váha = 0)} 
  \item Středně obecný \emph{(váha = 1)}
  \item Málo obecný \emph{(váha = 2)}
\end{list}

\subsection*{Ročník}
Pro případné budoucí úpravy systému, či filtrování, je vhodné u předmětu vyplnit i ročník, ve kterém je vyučován. S ohledem na opakování předmětů může každé entitě přiřazeno neomezené množství ročníků (i žádný). 

\begin{list}{-}{Hodnoty}
  \item 2012/13
  \item 2013/14
  \item 2014/15
\end{list}

\section{Téma vzhledu a jeho rozložení}
\label{sec:tema-vzhledu}
Pro rozvržení stránek bylo využito tématu vzhledu \emph{Omega} ve verzi 4 (sekce \ref{subsec:omega}). Díky využití \emph{SUSY} je rozvržení stránky implementována přímo v kódu a v administračním rozhraní jsou určovány až pozice jednotlivých bloků \emph{Drupalu} - každé téma vzhledu definuje množinu dostupných segmentů stránky, do kterých lze vkládat bloky s vlastním obsahem portálu. 

\begin{figure}[htp] 
  \centering{\includegraphics[width=12cm]{img/page-layout-crop.pdf}}
  \caption{Rozvržení stránky}
  \label{fig:rozvrzeni-stranky}
\end{figure}  

Jak lze vidět na obrázku \ref{fig:rozvrzeni-stranky}, je stránka rozdělena na čtyři hlavní segmenty - hlavičku, postranní panel, obsah a patičku. Pro možnosti budoucího rozšíření je navíc definován segment Postskript, který se nachází ve spodní části obsahové části stránky. Zatímco hlavička i patička zabírají celou dostupnou šíři stránky (960px), centrální segment je rozdělen mezi postranní panel a obsah - poměr mezi těmito segmenty je určen za použití knihovny \emph{SUSY} a \gls{mediaqueries} a mění se v závislosti na šíři okna prohlížeče\cite{responsible-drupal}. 

\begin{description}
  \item[Telefon] (velikost okna do 44em) \hfill \\
  Postranní panel zabírá celou šířku a za ním následuje obsah.
  \item[Tablet + standart] (velikost okna mezi 44em a 100em) \hfill \\
  Postranní panel a obsah jsou zobrazeny v poměru 1 : 3.
  \item[Širokoúhlý monitor] (velikost okna nad 100em) \hfill \\
  Postranní panel a obsah jsou zobrazeny v poměru 1 : 4.  
\end{description}

Tento způsob změn rozložení je vhodný pro využití při klasickém procházení stránkou. S ohledem na využítí \emph{Guacamole} a připojení se na vzdálenou pracovní plochu přímo z portálu bylo potřeba zajistit vhodnou strukturu html kódu stránky, který by typicky obsahoval veškerý generovaný kód. Pro oddělení těchto dvou nesourodých prostředí je využito \emph{Layouts} - různá rozložení stránky obsažená v tématu \emph{Omega}. Každé z těchto rozložení implementuje vlastní šablony (viz. dále) pro generování html kódu a tím umožňuje kompletně změnit výsledný kód. Při vývoji je potřeba respektovat určitá pravidla, jako je počet a pojmenování segmentů, které musí odpovídat tématu vzhledu, ve kterém budou využita. Přepínání mezi rozloženími stránky se definují za pomocí modulu \emph{Context} a jeho podmodulu \emph{Context Layouts} (sekce \ref{subsec:context}). Zatímco pro všechny stránky portálu je využito rozložení s názvem Blue (podle barvy vzhledu stránek), v případě že se jedná o stránku implementující připojení ke vzdálené ploše (identifikace probíhá na základě \gls{url}), systém přepne na rozložení Guacamole (název dle využité technologie).

Při vlastním vývoji vzhledu portálu byly využity technologie a rozhraní dostupné v \gls{cms} \emph{Drupal}, ve kterých probíhá vývoj ve třech úrovních.

\begin{enumerate}
  \item Je potřeba generovat vhodný HTML kód. K tomu slouží systém přepisování šablonovacích souborů (přípona .tpl.php) pro jednotlivé prvky webu. V základu je dostupná stromová struktura, začínající u celého dokumentu (html.tpl.php), který obsahuje veškerý obsah včetně hlavičky <head>, a pokračující přes stránku (page.tpl.php) až po jednotlivé elementy (např. fieldset.tpl.php). Většina těchto šablon je definována přímo v jádru systému a všechny lze jednoduchým způsobem (na základě jména souboru) přepsat. Zároveň je možné vytvářet úplně nové šablony, které lze následovně volat při vytváření nových elementů. Šablonám lze i upravit data, která do nich budou zaslána a to implementací funkcí s příponami \_preprocess a \_process a jejich vložením do zdrojového kódu. 
  
  \item Výstup html lze měnit přepsáním funkcí formátujících jednoduché elementy stránky (například seznam elementů ul). Typicky jsou tyto prvky generovány funkcí \texttt{theme([jméno elementu], [proměnné hodnoty])}, která může být přepsána v modulu nebo tématu vzhledu, jako [jméno modulu]\_[jméno elementu]() - například \texttt{function omega\_fieldset(\$variables)}. Téma vzhledu \emph{Omega} nad tímto rozhraním navíc umožňuje všechny tyto funkce vkládat do samostatných souborů pojmenovaných dle specifických konvencí, čímž dále zpřehledňuje výsledný kód (kód výše zmíněného příkladu by byl umístěn v souboru \texttt{theme/container.theme.inc}).
  \item Na HTML kód se nanášejí \gls{css} styly. Jak je popsáno v sekci \ref{subsec:sass}, je výhodnější využívat nástroj \emph{SASS}, který je do jazyka \gls{css} kompilován a poskytuje více možností. Vývoj probíhá dle standartů určených v tématu vzhledu \emph{Omega} (viz. \ref{subsec:omega}) a je definován v rámci několika \emph{SASS} souborů, každý z nichž obsahuje určitou skupinu nastavení.
\end{enumerate}

\begin{figure}[]
  \includegraphics[width=12cm]{img/drupal-theme-architecture-crop.pdf}
  \caption{Architektura vykreslení HTML kódu stránky}
  \label{fig:theme_architecture}
\end{figure}  

V repozitáři je již dostupná kompilovaná verze \gls{css} souborů, ale v případě nutnosti je možné je znovu vygenerovat. K tomu je nutno mít nainstalován programovací jazyk \emph{Ruby}, \emph{COMPASS} (sekce \ref{subsec:compass}) a \gls{bundler}. Kompilace se spustí následujícími příkazy:

\begin{lstlisting}[language=bash]
  $ sudo gem install bundler
  $ bundle install
  $ compass watch
\end{lstlisting}

\emph{COMPASS} bude od této chvíle sledovat obsah složky a v případě jakýchkoliv změn automaticky aktualizovat výsledné \emph{CSS} soubory.

\section{Proces přihlášení uživatelů}

Každý \gls{is} vytvořený v prostředí \gls{cms} \emph{Drupal} má automaticky vytvořeného uživatele s ID 1, který má všechna administrační práva. Kromě tohoto uživatele mohou být v systému vytvořy další uživatelské účty a každý z nich může být přiřazen k určité uživatelské roli (více o rolích v sekci \ref{sec:administrace}). Jelikož má být portál dostupný širší skupině uživatelů, jmenovitě studentům Masarykovy Univerzity, je vhodné využít pro autentizaci externí službu, v tomto případě se jedná o autentizaci skrze protokol Kerberos a server Masarykovy Univerzity. 

Uživatelé mohou pro své přihlášení využít své \gls{uco} a ihned mají přístup do systému. Je jim zobrazena nápověda, že pro přístup do \gls{aspi} je nutné zadat uživatelské jméno a heslo a zároveň jim je nabídnuta adresa \gls{url}, na které si o dané údaje mohou požádat. 

\section{Struktura zdrojových souborů}

\begin{itemize}
  \item \textbf{doc/} \hfill \\
    dokumentace projektu
  \begin{itemize}
    \item \textbf{mvantuch/} \hfill \\
      zdrojové kóody pro generování dokumentace projektu v 
    \item \textbf{omaterna/} \hfill \\
      práce Mgr. Ondřeje Materny určená pro referenci
  \end{itemize}
  \item \textbf{phing/} \hfill \\
    soubory určené k podpoře \gls{deployment} pomocí nástroje Phing
  \item \textbf{ref/} \hfill \\
    referenční soubory, design a pod.
  \item \textbf{src/} \hfill \\
    zdrojové kódy jednotlivých částí projektu
    \begin{itemize}
      \item \textbf{main/} \hfill \\
        aplikace Guacamole upravená pro potřeby tohoto projektu
      \item \textbf{modules/} \hfill \\
        moduly \gls{cms} Drupal vyvinuté speciálně pro tento projekt, nebo exportované za pomoci modulu Features (kapitola~\ref{subsec:features})
      \item \textbf{profiles/} \hfill \\
        instalační profil pro Drupal, definující veřejně dostupné moduly, potřebné k chodu
      \item \textbf{themes/} \hfill \\
       téma vzhledu projektu
    \end{itemize}
  \item \emph{build.xml} \hfill \\
    xml soubor obsahující instrukce pro nástroj Phing k operacím nad instalací Drupalu
  \item \emph{pom.xml} \hfill \\
    xml soubor s projektem Maven, kontrolujícím Guacamole část projektu
  \item \emph{esf.make} \hfill \\
    make skript pro příkaz Drush make
  \item \emph{INSTALL.md} \hfill \\
    markdown soubor s instalačními instrukcemi
  \item \emph{README.md} \hfill \\
    markdown soubor se všeobecnou nápovědou
\end{itemize}
