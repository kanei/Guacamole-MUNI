%% Requires fithesis2 module (can be downloaded from
%% https://github.com/arax/fithesis2
%% Load document class fithesis2
%% {10pt, 11pt, 12pt}
%% {draft, final}
%% {oneside, twoside}
%% {onecolumn, twocolumn}
%% sudo yum install texlive texlive-babel-czech texlive-hyphen-czech 
\documentclass[10pt,final,oneside]{fithesis2}

%% Basic packages
\usepackage[czech]{babel}
\usepackage{cmap}
\usepackage[T1]{fontenc}
\usepackage{lmodern}
\usepackage[utf8]{inputenc}
\usepackage{graphicx}

%% Additional packages for colors, advanced
%% formatting options, etc.
\usepackage{color}
\usepackage{microtype}
\usepackage{url}
\usepackage{cslatexquotes}
\usepackage{fancyvrb}
\usepackage[small,bf]{caption}
\usepackage[plainpages=false,pdfpagelabels,unicode]{hyperref}
\usepackage[all]{hypcap}
\usepackage{amssymb}
\usepackage{courier}
%% Fix long URLs in DVIs
\usepackage{ifpdf}

\ifpdf
\else
  \usepackage{breakurl}
\fi

%% Packages used to generate various lists
\usepackage{makeidx}
\makeindex

\usepackage[xindy]{glossaries}
\newacronym{ie}{IE}{Internet Explorer}

\newacronym{esf}{ESF}{Ekonomicko Správní Fakulta}

\newacronym{eu}{EU}{Evropská Unie}

\newacronym{cors}{CORS}{Cross-origin Resource Sharing}

\newglossaryentry{aspi}{
  name=ASPI, 
  description={,,Automatizovaný Systém Právních Informací'' - informační systém vyvíjený společností Wolters Kluwer, poskytující komplexní informace z právníckých oborů}
}

\newglossaryentry{api}{ 
  name=API, 
  description={Zkratka z anglického Application Programming Interface, definující rozhraní poskytované k integraci programů třetích stran}
}

\newglossaryentry{bundler} {
  name=Bundler, 
  description={}
}

\newglossaryentry{bash} {
  name=BASH, 
  description={}
}


\newglossaryentry{cms}{ 
  name=CMS, 
  description={Content Management System}
}

\newglossaryentry{css}{ 
  name=CSS, 
  description={}
}

\newglossaryentry{ci}{
  name=CI,
  description={Průběžná integrace (Continuous Integration) označuje souhrn nástrojů použitých k průběžné kontrole zdrojového kódu. Typicky sem patří spouštění testů, kontrola kvality kódu, statická analýza kódu a podobně.}
}

\newglossaryentry{cvs}{
  name=CVS,
  description={Systém ke správě verzí projektu (Concurrent Version System) slouží k ukládání historie verzí zdrojového kódu.}
}

\newglossaryentry{deployment}{
  name=nasazení,
  description={Proces instalace projektu na typicky vzdálený server a spuštění případných migračních skriptů a pododbně.}
}

\newglossaryentry{responsive}{
  name=responsivní web design, 
  description={Způsob stylování webových dokumentů, při kterém je brán ohled na různá rozlišení klientských zařízení (telefon, tablet, počítač)}
}

\newglossaryentry{wysiwyg}{
  name=WYSIWYG, 
  description={Zkratka výrazu ,,What you see is what you get'', doslowně přeloženo jako ,,dostaneš to co vidíš''. Používá se pro označení editorů html kódu, které poskytují formátování pomocí tlačítek a výstup automaticky konvertují do html kódu}
}

\newglossaryentry{url}{
  name=URL,
  description={}
}

\newglossaryentry{ruby} {
  name=Ruby, 
  description={}
}

\newglossaryentry{pear} {
  name=PEAR, 
  description={}
}

\newglossaryentry{curl} {
  name=CURL, 
  description={}
}

\newglossaryentry{composer} {
  name=Composer, 
  description={}
}

\newglossaryentry{xss} {
  name=XSS, 
  description={Cross Site Scripting, technika ...}
}

\newglossaryentry{session} {
  name=relace,
  description={Označuje přetrvávající spojení mezi serverem a klientem}
}

\makeglossaries

%% Use STAR and CIRCLE signs for nested
%% itemized lists
\renewcommand{\labelitemii}{$\star$}
\renewcommand{\labelitemiii}{$\circ$}

%% Title page information
\thesistitle{Vytvoření specializovaného klienta pro podporu e-learningového portálu}
\thesissubtitle{Diplomová práce}
\thesisstudent{Marek Vantuch}
\thesiswoman{false} %% Important when using Slovak or Czech lang
\thesisfaculty{fi}  %% {fi, eco, law, sci, fsps, phil, ped, med, fss}
\thesislang{cs}     %% {en, sk, cs}
\thesisyear{2014}
\thesisadvisor{Ing. Leonard Walletzký, Ph.D}

%% Beginning of the document
\begin{document}

%% Front page with a logo and basic thesis information
\FrontMatter
\ThesisTitlePage

%% Thesis declaration (required)
\begin{ThesisDeclaration}
  \DeclarationText
  \AdvisorName
\end{ThesisDeclaration}

%%\chapter*{Zadání práce}
%%Na základě analýzy možných Open Source klientů, podporujících připojení ke vzdálené ploše (Remonte Desktop) vyberte nejvhodnější a ten dále upravte tak, aby sloužil ke spuštění vzdálené aplikace po kliknutí na speciální odkaz v prohlížeči (aplikace ASPI). V případě nutnosti vytvořte obslužný plug-in do prohlížeče, sloužící pro předávání parametrů vzdálené aplikaci. Dále připravte obslužný systém pro vkládání textů, obsahujících odkazy do aplikace ASPI a jejich editaci.


%%TODO Thanks (optional)
\begin{ThesisThanks}
Chtěl bych obzvláště poděkovat Ing. Leonardu Walletzkému, Ph.D za příležitost pracovat na tomto projektu. Děkuji všem svým spolupracovníkům za ohleduplnost a trpělivost nutnou s ohledem na ztížené podmínky, které byly způsobeny mou prací ze zahraničí.
\end{ThesisThanks}

%%TODO Abstract (required)
\begin{ThesisAbstract}
Tato práce popisuje implementaci internetového portálu pro fakultu Ekonomicko-správní Masarykovy Univerzity. Jsou zahrnuty dvě hlavní části, popis úprav existujícího řešení postaveného na open-source CMS systému Drupal a jeho propojení pomocí vzdálené plochy na servery ASPI. Při vývoji byl kladen důraz na udržitelnost řešení, jeho strukturalizaci. Ačkoliv Drupal typicky udržuje všechna nastavení v databázi, pomocí dostupných nástrojů byla tato exportována do PHP kódu a byl nastaven proces jejich přenesení do testovacího či produkčního prostředí. Pro ulehčení tohoto procesu byly vytvořeny skripty a ty byly propojeny s ,,Continous Integration'' serverem. Důležitá byla i uživatelská přívětivost a jednoduchost, kterých bylo dosaženo sjednocením platformy postavené na Drupalu a existující implementace vzdálené plochy využívající HTML5 a JavaScriptu.
\end{ThesisAbstract}

%% Keywords (required)git
\begin{ThesisKeyWords}
Guacamole, vzdálená plocha, ASPI, ESF, Drupal, Team City, Drush, Phing, CORS, WebSockets, JAVA, PHP, JavaScript
\end{ThesisKeyWords}

%% Beginning of the thesis itself
\MainMatter

%% TOC (required)
\tableofcontents

%% Thesis text structured using
%% chapters, sections, subsections, etc.
\chapter{Úvod}
Tato práce se zabývá projektem, jehož cílem je vytvoření informačního systému pro studenty Ekonomicko-správní fakulty Masarykovy Univerzity. Stávající řešení v podobě webové stránka bylo postaveno na \gls{opensource} projektu \emph{Drupal}\footnote{https://drupal.org} verze 6 a celá jeho stuktura byla řešena na bázi slovníků a stránek, bez rozlišení typů obsahu. Mgr. Ondřej Materna ve své práci zanalyzoval možnosti zlepšení řešení a reálné požadavky studentů. Výsledkem je návrh řešení, který se neslučoval s existujícími stránkami, které tak musely být razantně přepracovány. \emph{Drupal} verze 6 existoval již více než pět let a dva roky existuje i jeho aktualizovaná verze 7\cite{website:wiki:drupal}. Ten poskytuje vyšší rychlost, stabilitu i rozšíření díky širší podpoře komunity a nezávislých vývojářů. Stávající řešení je hlouběji rozebráno v kapitole~\ref{chap:analyza}, ve které jsou zároveň popsány technologie využité k implementaci nové verze portálu a základní architektura řešení.

Důležitou částí portálu je propojení s \gls{aspi} za pomoci klienta vzdálené plochy. Ve stávajícím řešení byli studenti nuceni používat jednu ze dvou možností připojení:

\begin{enumerate}
  \item lokální instalace nativní aplikace \gls{aspi} na klientský počítač a  její spuštění - odkazy se otevírají přímo v aplikaci
  \item připojení se na vzdálenou plochu pomocí jednoho z veřejně dostupných klientů a používání nativní aplikace zde
\end{enumerate}

Odkazy na stránkách se však automaticky nepřenášely na vzdálenou plochu a celkově vyžadovalo toto řešení vyšší technickou zdatnost uživatelů.

Hlavním cílem projektu je uživatelská přívětivost a proto byl při výběru řešení kladen důraz převážně na jednoduchost a minimální požadavky na klientské zařízení a uživatele. Bylo zvoleno řešení postavené na prvcích jazyka HTML5 a open-source nástroji \emph{Guacamole}. Ten poskytuje klienta vzdálené plochy čistě skrze okno prohlížeče. Komunikace se vzdálenou plochou probíhá za pomoci sprostředkovatelského proxy serveru. Architektura je detailně popsána v kapitole \ref{chap:implementace-guacamole}.

Jak bylo zmíněno výše, tato práce staví na diplomové práci Mgr. Ondřeje Materny ,,Návrh a realizace právního portálu pro ESF MU''\cite{omaterna2013}. Její obsah je důkladně zanalyzován a z ní vyplývající poznatky uplatněny na prostředí \gls{cms} \emph{Drupal} a jeho možnosti. Proces aktualizace a implementace funkční a základ vzhledové stránky řešení jsou popsány v kapitole  \ref{chap:implementace-drupal}. Vzhled samotný není předmětem této práce, nýbrž práce Bc. Ivany Haraslínové. 

Z důvodu spolupráce mezi více studenty byl vytvořen základní systém pro vedení projektu, využívající prostředí systému \emph{GitHub} a jeho stručný popis je obsažen v kapitole~\ref{chap:vyvoj}. Projekt je veřejně dostupný na adrese \url{https://github.com/kanei/esf-mu-portal}.

\chapter{Analýza}
\label{chap:analyza}
Jak byly požadavky zpracovány a navrženy \\
Jakým způsobem bylo vytvořeno propojení na vzdálený server \\
Při analýze byl velký důraz kladen na jednoduchost administrace celého projektu. K tomu bylo využito co největšího omezení možností administrátorů a namísto toho integrace logiky vztahů do nastavení celého projektu.

\section{Existující řešení}
Při analýze bylo nutné zohlednit stávající řešení a možnosti jeho rozšíření. Existující portál využíval Drupal v 6. verzi, který byl již v dané době značně zastaralý a jeho rozšiřitelnost byla velmi omezená, či zbytečně komplikovaná. Řešení bylo rozšířeno řadou volně dostupných modulů, které dohromady poskytovaly komplexní strukturu nabídky menu vytvořenou namíru požadavkům fakulty. Tato struktura byla navržena v době, kdy nebylo známo, že systém bude pro zobrazování zákonů využívat externí software ASPI a proto byly tyto zákony také použity. 

[Popsat strukturu menu]

\section{Použité technologie}
\label{sec:technologies}

Základními technologiemi jsou Drupal a Guacamole, které prakticky určují, jaké další technologie byly využity. Základem proto jsou programovaí jazyky PHP, JAVA a JavaScript a výstup je realizován pomocí HTML (v.4 a v.5) a CSS (v.2 a v.3). Ačkoliv bylo využito velké množství modulů, rozhodl jsem se zmínit pouze ty, které měly zásadní vliv buď na funkcionalitu, nebo na proces vývoje.

\subsection{Jádro CMS a přidružené technologie}

\subsubsection*{Drupal \hfill \emph{http://drupal.org}} 
Aktuálně světově třetí nejrozšířenější\cite{website:cms-market-share} systémů pro správu obsahu (CMS), Drupal je založený na jazyce PHP klade důraz obzvláště na vývojáře a na možnosti úpravy stránek. Zatímco WordPress cílí na uživatelskou jednoduchost a většina stránek na něm postavných je lehce rozpoznatelná, Drupal je možné změnit od základu a jeho \gls{api} pro tvorbu modulů poskytuje jednoduché možnosti úprav. Systém je postavený na PHP, minimálně verze 5.3 a i když některé jeho části jsou již implementovány objektově, celkově převládá funkcionální přístup s množstvím propriétárních principů. Mezi ty patří například hook\_api, poskytující přípojné body pro další moduly a tím jednoduchou rozšiřitelnost, nebo systém vzorů (templates), které pomocí speciální jmenné konvence umožňují měnit výpis html kódu prvků webu.

\subsubsection*{Omega (Drupal Theme) \hfill \emph{http://drupal.org/project/omega}}
Pro Drupal existuje nespočet témat vzhledu, které se starají o formátování výstupu html kódu a s ním spojených kaskádových stylů (css). Projekt Omega je zaměřen na \gls{responsive} a poskytuje hotovou implementaci různých stylů pro různá zobrazovací zařízení. Zatímco třetí verze poskytuje komplexní administrační rozhraní, přístup čtvrté verze se spíše zaměřil na implementaci v kódu. Za pomocí SUSY jsou zapsány poměry mezi bloky zobrazenými na stránce a ty se pak při využití knihovny Breakpoints\cite{website:breakpoints} mezi sebou přepínají.

\subsubsection*{SASS - Syntactically Awesome Style Sheets \hfill \emph{http://sass-lang.com/}}
Při využití tématu vzhledu Omega se automaticky nabídla možnost využití knihovny SASS pro generování kaskádových stylů. Namísto použití klasického CSS je kód napsán ve formátu SCSS a poté zkompilován do CSS. Tento přístup přináší bezproblémovou kompatibilitu se všemi prohlížeči doplněnou o rozšířené možnosti definice pravidel, jako je využití proměnných, vnoření pravidel, in-line import a další. Kromě výrazného vylepšení čitelnosti, možnosti seskupováni pravidel bez zvyšování zátěže na přenos dat tak lehce lze dosáhnout i snížení programátorské náročnosti.

\subsubsection*{COMPASS \hfill \emph{http://compass-style.org/}}
Soubory SASS musí být kompilovány do CSS pro zobrazení prohlížečem a protože manuální kompilace je časově náročná a náchylná k opomenutí, nabízí se využití programu COMPASS. Přestože poskytuje širokou škálu funkcionality určené pro ulehčení práce designérům, pro potřeby tohoto projektu byla využita pouze automatická konverze z SCSS do CSS dle definovaných pravidel za pomocí compass watch. Po spuštění démona jsou kontrolovány všechny změny ve složce a automaticky regenerovány výstupní CSS soubory pro načtení prohlížečem.

\subsubsection*{SUSY \hfill \emph{http://susy.oddbird.net/}}
Další technologií využitou v tématu vzhledu je SUSY - implementace responsivní mřížky pro Compass. Za pomocí jednoduchých pravidel lze nadefinovat rozdílné rozvržení stránky závisející na rozlišení zobrazovacího zařízení. Například monitoru počítače s rozlišením vyšším než 1024 bodů můžeme postranní panel zobrazit nalevo, zatímco na mobilním zařízením můžeme text zmenšit a zobrazit v horní části stránky spolu s vypuštěním některých nedůležitých bloků. Celá stránka je může být rozdělena na počet sloupců, které mohou být dynamicky vyplňovány a pravidla lze velmi jednoduše zapisovat bez nutnosti řešení problémů s kompatibilitou mezi prohlížeči.

\subsection{Technologie připojení ke vzdálené ploše}

\subsubsection*{Guacamole \hfill \emph{http://guac-dev.org/}}
Jak je popsáno na stránkách této knihovny, jedná se o implementaci myšlenky o připojení ke vzdálené ploše skrze webový prohlížeč. Namísto potřeby instalace speciálního klienta, či využití existujícího systémového klienta, je možné používat jakýkoliv prohlížeč podporující některé z technologií HTML5, jako je canvas. Připojení probíhá skrze proxy server implementovaný v jazyce C, který zprostředkovává vlastní připojení. Ke klientovi putují již jen obrazová data a povely, obojí je zakódováno v proprietárním protokolu umožňujícím snížit odezvu obrazu a dekódovat jej v průběhu přijímání dat. Postup komunikace mezi klientským prohlížečem a vzdálenou plochou je blíže vyobrazen na diagramu \ref{fig:arch_core}.

Propojení mezi webovým prohlížečem a proxy serverem je provedeno za pomocí standatních AJAX požadavků v jazyce JavaScript, zatímco propojení z proxy dále podporuje protokoly SSH, VNC a RDP.

\begin{figure}[htp] 
  \centering{\includegraphics[scale=0.77]{img/architecture-core-crop.pdf}}
  \caption{Základní architektura komunikace se vzdálenou plochou}
  \label{fig:arch_core}
\end{figure}  

\subsubsection*{CORS} 
Kvůli bezpečnostním rizikům je JavaScriptu běžícímu v prohlížeči zakázáno přistupovat k jiným doménám, tento přístup se nazývá Same-origin policy a slouží k zamezení podvodných stránek a zvýšení internetové bezpečnosti. Veškeré požadavky směřující na jinou doménu tak musejí být vykonávány přímo ze serveru, což je často zbytečně náročné a v dnešní době je občas nutné tato omezení obejít. K tomu slouží CORS - Cross-origin Resource Sharing, kdy za pomocí HTML hlaviček přidaných do komumnikace mezi serverem a prohlížečem je umožněna komunikace s jinými doménami. Pokud server odešle zpět hlavičku Access-Control-Allow-Origin: [doména], prohlížeč ji zanalyzuje a případně umožní komunikaci. K tomuto je nutné využít XMLHttpRequest (XDomainRequest v případě \gls{ie}) s attributem WithCredentials, o zbytek se postará prohlížeč se serverem a komunikace funguje stejně jako v případě klasických požadavků.

\subsubsection*{ASPI \hfill \emph{http://www.systemaspi.cz/}}

\subsection{Podpora deploymentu a vývoje}

\subsubsection*{Drush \hfill \emph{http://github.com/drush-ops/drush}}
Pro ulehčení administračních úkonů nad instalací Drupalu byl komunitou vyvinut program Drush, poskytující administrační rozhraní nad Drupalem v terminálové konzoli. V základu jsou poskytovány operace nad moduly a tématy vzhledu, kdy každý z modulů může pomocí implementace funkcí v souboru .drush.inc poskytnout vlastní příkazy.

\subsubsection*{Phing \hfill \emph{http://www.phing.info/}}
Phing je nástroj určený ke kompilaci PHP projektů, fungující na XML konfiguračním souboru. Je založený na technologii Apache ANT poskytující stejný nástroj pro jazyk JAVA. V konfiguračním souboru lze definovat pravidla, proměnné a závislosti vedoucí k doručení projektu, jeho kontrolám, či automatizaci jakýchkoliv akcí. 

\subsubsection*{DrushTask \hfill \emph{https://drupal.org/project/phingdrushtask}}

\subsubsection*{Maven}

\subsubsection*{Team City}

\subsubsection*{GitHub}
Pro udržování historie kódu a pro spolupráci více autorů byla vybrána technologie Git a online portál GitHub, který pro ni poskytuje jednoduchou instalaci a online správu uložiště. Všechny úpravy mohou být jednoduše zobrazeny přímo v prohlížečí a provázány s úkoly vytvořenými v úkolovém managementu na stejném internetovém portálu. 

\section{Architektura připojení ke vzdálené ploše}
Základně poskytuje Guacamole komplexní řešení na propojení ke vzdálené ploše včetně stránky pro management uživatelů, kde každý z nich si může pro sebe vytvořit své vlastní připojení, u každého z nich se zobrazuje náhled a podobně. Jak je zobrazeno na diagramu \ref{fig:arch_guacamole}, součástí Guacamole je i webová aplikace v jazyce HTML, která implementuje knihovnu Guacamole-js a tím i komunikaci s JAVA Servletem. Pro potřeby tohoto projektu však bylo nutné tyto JavaScriptové knihovny převzít a implementovat do modulu pro Drupal. 

\begin{figure}[htp] 
  \centering{\includegraphics[scale=0.7]{img/architecture-guacamole-crop.pdf}}
  \caption{Neupravená architektura a rozčlenění řešení}
  \label{fig:arch_guacamole}
\end{figure}  

Díky skutečnosti, že již nutně neběží JAVA Server a JAVA EE Web aplikace na stejném serveru, či url, bylo nutné změnit architekturu AJAXových požadavků. Ty totiž musejí v typickém případě být zasílány pouze na stejnou doménu kvůli prevenci \gls{xss}. Pro vyřešení tohoto problému byla změněna implementace na CORS požadavky, k čemuž musel být uzpůsoben i JAVA Servlet kontejner, u kterého musely být povoleny požadavky z dané domény. 

Uživatel proto pracuje základně pouze s portálem postaveném na Drupalu, do kterého byl přidán modul implementující stránku určenou ke komunikaci s Guacamole Servlet kontejnerem za pomoci upravené JavaScriptové knihovny Guacamole-js, jak je vyobrazeno na diagramu \ref{fig:arch_drupal}.

\begin{figure}[htp] 
  \centering{\includegraphics[scale=0.7]{img/architecture-drupal-crop.pdf}}
  \caption{Architektura přizpůsobená pro portál ESF}
  \label{fig:arch_drupal}
\end{figure}  

\section{Struktura portálu v prostředí Drupalu}

V Drupalu je využíváno několika základních ,,stavebních kamenů'', každý z nich může dle svého typu hluboce modifikován a přizpůsoben každému řešení. Základem jsou entity, uložené v databázi poskytují možnosti přidávání polí, podobně jako sloupců v databázi. Entity jsou několika moduly jádra Drupalu rozšířeny a poskytují funkcionalitu potřebnou k vytvoření webových stránek. Jmenovitě se jedná o uzly (node), které nám poskytují jednotlivé stránky a mají svou unikátní url adresu, a slovníky (taxonomy), které slouží ke kategorizaci obsahu a vytváření polí založených na výběru z dostupných prvků. Uzly i slovníky mohou být konfigurovány ve formě druhů obsahu, každý z nich s vlastními polemi nezávislými na ostatních entitách. Pole se stejným obsahem však mohou být sdílena pro ulehčení databázové struktury, do ktére se data reálně ukládají. Instance dostupných typů obsahu pak dostanou přidělené unikátní číslo v rámci všech entit, které určuje její data v databázi. Ne všechna pole musejí být zobrazena na stránce entity, nebo může pro jejich zobrazení být využito speciálních formátovacích technik, jako tabulek, či jiných výpisů.

Pro umístění obsahu na stránkách se využívají bloky (block), které lze přímo v UI umístit do předdefinovaných segmentů stránek (ty jsou definované v konfiguračních souborech tématu vzhledu viz. kapitola \ref{chap:implementace-drupal}).

Požadavky \cite{omaterna2013} poskytnuté Mgr. Maternou poskytují analýzu řešení, kterou však bylo potřeba aplikovat přímo na možnosti Drupalu a co nejvíce využít jeho možností - jednak pro samotnou implementaci řešení, ale také pro maximální ulehčení administrace a poskytnutí co nejvyšší míry soudržnosti skrze finální řešení projektu. Toho je dosáhnuto omezením vstupu a poskytnutím výběru z předdefinovaných možností (pokud je tento způsob řešení možný).

\subsection{Typy polí použitelné pro vlastnosti typů obsahu}
V drupalu existuje několik základních typů polí, které převážně fungují podobně jako datové typy v typických programovacích jazycích. Narozdíl od jednoduchých datových typů však poskytují navíc html formátování a mohou poskytovat i další nastavení. Formátování jde dále změnit pomocí přepsání základních formátovácích funkcí.

\subsubsection*{Text}
Poskytuje pole pro text s omezenou délkou bez možnosti vkládání html značek. Jeho obsah je zadáván pouze v jediném řádku a poskytuje pouze limitované možnosti nastavení, kterými je délka vstupního pole, či defaultní hodnota. 

\subsubsection*{Dlouhý text se souhrnem}
Narozdíl od typu pole Text může v tomto případě být text neomezeně dlouhý a obsahovat značky html kódu. Ty mohou být vkládány buď ručně a nebo automaticky formátovány pomocí \gls{wysiwyg} editoru. Zároveň je pro text možné vyplnit souhrn, který je ve specifických případech zobrazit, nebo se automaticky zobrazí zkrácená verze textu. Počet znaků je možné určit pro každou instanci pole zvlášť.

\subsubsection*{Hodnota slovníku}
V nastavení pole lze vybrat jeden z existujících slovníků. Jeho hodnoty pak jsou vypsány v jednom z dostupných formátů (radio/zaškrtávací tlačítka) a dle nastavení pak lze vybrat jen jedna, nebo více možností.

\subsubsection*{Logická hodnota}
Poskytuje pole pro logickou hodnotu - ano nebo ne. Pro hodnoty lze vybrat jednu z typicky použítelných nápisů (ano/ne, pravda/nepravda, checkmark/x), nebo definovat nápisy vlastní. 

\subsubsection*{URL adresa}
Kromě samotné adresy poskytuje možnost uložení dalších parametrů, podobně jako element <a> jazyka html. Mezi tyto parametry patří např. titulek, cíl a podobně.

\subsubsection*{Média}
Pro vkládání médií do systému je využit modul Media, který přidává komplexní řešení pro vkládání jakéhokoliv obsahu. Je možné vkládat dokumenty, soubory pdf, zvuk, obraz, či video a všechny tyto typy obsahu jsou uživately poskytovány v jednotném formátu.

\subsubsection*{Váha}
I když by se pro určení váhy produktu dalo využít jednoduchého číselného pole, jeden z rozšiřujících modulů poskytuje jednak speciálně využitelné pole pro výběr hodnoty a také pohledy na obsah a řadící funkce, které celou implementaci značně ulehčují.

\subsubsection*{Odkaz na entitu}
Důležitou vlastností většiny systémů je možnost propojení entit. Pro modelování těchto vztahů zde existuje typ pole odkaz na entitu, u kterého je možné vybrat typy entity, na které je daný typ obsahu navázán a způsob propojení. Za pomocí extra modulů lze také možné modelovat místo jednoduché asociace kompozici. V tom případě při smazání nadentity jsou smazány i všechny podentity. Lze také definovat, zda je možné pouze vytvářet nové podentity, či přidávat ty již existující.

\subsection{Typy obsahu}
Kdybychom si představili typy obsahu ve světě objektově orientovaného programování, všechny by dědily vlastnosti ze základní třídy \emph{Entita}. Těmito vlastnostmi jsou typ, ID a několik dalších důležitých pouze pro interní fungování redakčního systému. Z \emph{Entity} dědí třída \emph{Uzel}, která ji rozšiřuje o nadpis, URL a stav. Pro lepší představu jsou vztahy popsány na diagramu \ref{fig:class-diagram}. Přímý předek typu obsahu je vždy zapsán v závorce za jeho názvem.

\begin{figure}[htp] \centering{
\includegraphics[scale=0.95]{img/class-diagram-crop.pdf}}
\caption{Diagram tříd CMS systému}
\label{fig:class-diagram}
\end{figure}  

\subsubsection*{Předmět \emph{(uzel)}}
Popisuje předmět vyučovaný v rámci fakulty ESF a tím vyjadřuje základní rozdělení celého portálu a slučuje hlavní část informací, o které se návštěvníci zajímají. Na stránce detailu předmětu je zobrazen ktátký popis, za kterým následuje tabulka se seznamem studijních textů a k nim vztahujících se materiálů. Materiály, které se vztahují přímo k předmětu jsou zobrazeny samostatně a celá stránka je uzavřana seznamem oborů, které se k danému předmětu vztahují.

\subsubsection*{Obor práva \emph{(uzel)}}
Předměty se mohou vztahovat k určitým oborům práva. Zatímco některé z nich se mohou dotýkat pouze jednoho z nich, jiné se mohou dotýkat všech. Proto je důležité tento vztah být vymodelovat a uživatelům poskytnout možnost se v tomto jednoduše orientovat a přecházet mezi obsahem, který jej právě zajímá. Na stránce oboru práva je zobrazen popis oboru a tabulka se seznamem předmětů, které se k němu vztahují.

\subsubsection*{Studijní text \emph{(položka)}}
Studijní text obsahuje informace vztahující se k jednomu celku výuky, typicky k jedné přednášce. Lze definovat zkrácený cíl přednášky, přidat k ní materiály a určit její pořadové číslo. U všech materiálů je vhodné zadat název pro správné zobrazení v tabulce detailu předmětu, na kterém se zobrazuje seznam těchto entit.

\subsubsection*{Zákon \emph{(položka)}}


\subsubsection*{Příběh \emph{(uzel)}}
Pro odreagování studentů je v jednom ze segmentů stránek zobrazován příběh z právního prostředí. 

\subsubsection*{Vtip \emph{(uzel)}}
V postranním panelu lze zobrazit vždy aktuální vtip z právního prostředí, který také slouží k odreagování studentů. \\

\subsubsection*{Novinka \emph{(uzel)}}
Na stránkách jsou publikovány novinky (aktuality) týkající se buď portálu samotného, fakulty ESF, či jakýchkoliv dalších relevantních událostí. Novinky jsou zobrazovány na úvodní stránce. \\

\subsection{Slovníky}

Pro kategorizaci a pro výběr možností bylo v požadavcích identifikováno několik slovníků. Díky vztahu k entitě (slovník také dědí vlastnosti entity) může také pro své hodnoty definovat pole a tím rozšířit jejich funkcionalitu a množství informací.

\subsubsection*{Stupeň obecnosti}
Pro řazení oborů je potřeba uložit jejich obecnost, která je vyjádřena v několika stupních, které mohou být v případě potřeby rozšířeny. Každé entitě může být přiřazen jeden stupeň obecnosti.\\

\begin{list}{-}{Pole}
  \item Váha (váha) \\
    číselná hodnota použitelná k řazení
\end{list}

\begin{list}{-}{Hodnoty}
  \item Velmi obecný \emph{(váha = 0)} 
  \item Středně obecný \emph{(váha = 1)}
  \item Málo obecný \emph{(váha = 2)}
\end{list}

\subsubsection*{Ročník}
Pro případné budoucí úpravy systému, či filtrování, je vhodné u předmětu vyplnit i ročník, ve kterém je vyučován. S ohledem na opakování předmětů může každé entitě přiřazeno neomezené množství ročníků (i žádný). \\

\begin{list}{-}{Hodnoty}
  \item 2012/13
  \item 2013/14
  \item 2014/15
\end{list}

\chapter{Implementace portálu v Drupalu}
\label{chap:implementace-drupal}

Pro převedení stávajícího řešení na novou verzi bylo nutné vykonat několik kroků. Nejprve byla data z produkčního prostředí přesunuta na testovací server, kde bylo řešení zanalyzováno co se týče implementační stránky, neboť bylo možné k portálu přistupovat s administrátorským účtěm bez jakýchkoliv omezení. Jakmile byly stránky zanalyzovány, byl započat proces jejich aktualizace na novou verzi Drupalu při zachování veškerého obsahu. Tento postup se ukázal jako kontraproduktivní, neboť i když byl obsah stránek převeden vpořádku, bylo potřeba jej převést na kompletně odlišnou strukturu spolu s jinými typy obsahu a jinými cílovými poli. Proto bylo rozhodnuto, že data budou přesunuta ručně.

\section{Aktualizace Drupal 6 na Drupal 7}
Výhody Drupal 7 nad Drupalem 6

\begin{table}
  \caption{Porovnání verzí modulů mezi Drupalem 6 a 7}
  \begin{tabular}{ | p{5cm} | p{2.5cm} | p{2.5cm} | c | }
    \hline 
    Jméno modulu & Drupal 6 & Drupal 7 & stav  \\ \hline 
    Backup and Migrate & 2.7 & 2.7 & \checkmark \\ \hline
    Colorbox & 1.6 & 2.4 & \checkmark \\ \hline
    CCK & 2.9 & jádro & \checkmark \\ \hline
    Custom Breadcumbs & 1.5 & 1.x-alpha3 & \\ \hline
    DB Maintenance & 1.4 & 1.1 & \checkmark \\ \hline
    DHTML Menu & 3.5 & 1.0-beta1 & \\ \hline 
    Email Field & 1.4 & 1.2 & \checkmark \\ \hline
    File (Field Paths & 1.5 & 1.0-beta4 & \\ \hline
    FileField & 3.11 & jádro & \checkmark \\ \hline
    Front Page & 1.3 & 2.4 & \checkmark \\ \hline
    ImageAPI & 3.11 & jádro & \checkmark \\ \hline
    ImageCache & 2.0-rc1 & jádro & \checkmark \\ \hline
    IMCE & 2.5 & 1.7 & \checkmark \\ \hline
    IMCE Wysiwyg bridge & 1.1 & 1.0 & \checkmark \\ \hline
    jCarousel & 2.6 & 2.6 & \checkmark \\ \hline
    Link & 2.10 & 1.1 & \checkmark \\ \hline
    Localization Update & 2.10 & 1.1 & \checkmark \\ \hline
    Menu Attributes & 2.0-beta1 & 1.0-rc2 & \checkmark \\ \hline
    Menu Block & 2.4 & 2.3 & \checkmark \\ \hline
    Pathauto & 1.6 & 1.2 & \checkmark \\ \hline
    Taxonomy Breadcrumb & 1.1 & - & \\ \hline
    Token & 1.19 & 1.5 & \checkmark \\ \hline
    Transliteration & 3.1 & 3.1 & \checkmark \\ \hline
    Views & 2.16 & 3.7 & \\ \hline
    Views Search & 1.0 & - & \\ \hline
    WYSIWYG & 2.4 & 2.2 & \checkmark \\ \hline
  \end{tabular}
\end{table}

\subsection{Odebrané moduly}
Funkcionalita některých z modulů byla v sedmé verzi přesunuta přímo do jádra Drupalu a proto nebylo jejich využití již potřebné - byly využity jejich ekvivalenty z jádra. Kromě těchto byly odebrány i další moduly, kdy některé nejsou pro novou verzi vůbec dostupné. Dále byly odstraněny moduly, jejichž využití již postrádalo smysl. Z důvodu změny a zjednodušení struktury menu již nebylo potřeba mnoho modulů, které umožňovaly složitá menu zobrazovat uživatelům. Zároveň bylo odstraněno několik modulů, které nebyly na stránkách vůbec využity a zbytečně zpomalovaly běh portálu. Následuje seznam těchto modulů s jejich krátkým popisem.

\subsubsection*{Nedostupné moduly}
\begin{itemize}
  \item Views Search
\end{itemize}

\subsubsection*{Moduly poskytující strukturu předchozího řešení}
\begin{itemize}
  \item Front Page - Slouží k nastavení výchozí stránky pro různé uživatelské role. Z důvodu plánování kompletně nové implementace domovské stránky byla tato funkcionalita prozatím odstraněna a může být případně přidána v čase implementace dané části portálu.
  \item Menu Attributes - Poskytuje %TODO doplnit
  \item Taxonomy Breadcrumb
  \item Custom Breadcrumbs
  \item DHTML Menu
\end{itemize}

\subsubsection*{Nepotřebné moduly}
\begin{itemize}
  \item jCarousel - Slouží k propojení se stejnojmennou JavaScriptovou knihovnou poskytující zobrazení obrázků v ovladatelném ,,kolotoči''. S ohledem na aktuální absenci jakékoliv grafiky byl modul shledán nepotřebným. 
  \item Colorbox - Podobně jako jCarousel slouží k propojení s knihovnou určenou ke zobrazování obrázků a ani tento modul nebyl aktivně využíván.
\end{itemize}

\subsection{Nově přidané moduly a jejich popis}

\subsubsection*{Features} 
Jak je popsáno v kapitole~\ref{chap:vyvoj}, Features slouží k ulehčení vývoje a správy nastavení stránek postavených na Drupalu. Základní myšlenkou je třívrstvé rozvržení systému, v první úrovni je jádro a jeho moduly, ve druhé rozšiřující moduly (tzv. Contrib) a ve třetí jsou rysy (Features), které v sobě zapouzdřují samotné nastavení stránek. Ačkoliv se jedná také o moduly, jsou generovány systémem na stránce \texttt{admin/structure/features}. Zde je možné buď vytvořit nový rys, či spravovat ty stávající. Ty mohou být v několika stavech závisejících na aktuálním nastavení systému. Pokud bylo v sytému něco změněno od chvíle, kdy byl rys vygenerován a povolen (případně pouze povolen, pokud byl přenesen z jiného serveru), je možné dané změny buď vrátit ke stavu definovanému v rysu, nebo rys aktualizovat na novou verzi.

\subsubsection*{Features Extra}
Jelikož modul Features v sobě implementuje pouze základní funkcionalitu exportu typů obsahu a poskytuje \gls{api} pro další moduly, některé prvky webu nejsou v základu exportovatelné. Tento modul možnosti exportu rozšiřuje o několik dalších možností, ze kterých pro tento projekt byla důležitá možnost exportovat nastavení umístění bloků.

\subsubsection*{Media}


\subsubsection*{Workbench}
\subsubsection*{Workbench Media}

\section{Téma vzhledu a jeho rozložení}
Pro rozvržení stránek bylo využito tématu vzhledu Omega ve verzi 4. Narozdíl od předchozích verzí, které poskytovaly uživatelům komplexní administrační rozhraní pro nastavení rozvržení stránky, byla tato možnost z verze 4 odstraněna a nahrazena využitím knihovny SUSY (viz. sekce~\ref{sec:technologies}). Veškerá rozvržení jsou proto implementována přímo v kódu a v administračním rozhraní jsou určovány až pozice jednotlivých bloků Drupalu. 
Jak lze vidět na obrázku \ref{fig:rozvrzeni-stranky}, je stránka rozdělena na ...

\section{Deployment a jeho možnosti}

\section{Optimalizace řešení a její dopad na uživatelskou zkušenost}
Ideálně si projít nějaké výzkumy Googlu ohledně odezvy a jejího dopadu na uživatele
Popsat agregaci JS, CSS a podobne 
\cite{website:drupal:optimizing}

\chapter{Implementace připojení ke vzdálené ploše}
\label{chap:implementace-guacamole}
Implementace připojení ke vzdálené ploše probíhala v několika fázích. Nejdříve bylo potřeba spustit řešení Guacamole lokálně a ujistit se, že je použitelné pro naše potřeby. Bylo také nutné analyzovat, jaké všechny možnosti poskytuje a které vlastnosti jsou potřebné a které přebytečné. Jako přebytečné byly idnetifikovány vlastnosti poskytující správu uživatelů, připojení a jejich skupin. Dále je implementována podpora klávesnici na obrazovce, ukládání snímků obrazovky a mnoho dalších rozšířených možností, které nebyly do základní verze vyžadovány a byly spíše kontraproduktivní s ohledem na zvýšení komplexnosti kódu a debugování celého řešení. 

Jakmile byla funkčnost ověřena, bylo nutné celou klientskou část řešení přesunout do modulu v Drupalu a upravit pro tamější podmínky. Bylo nutné změnit HTML výstup, kdy Drupal si v základu buduje strukturu pomocí několika šablonových souborů pro stránku, tělo a pod. Bylo nutné přidat JavaScriptové knihovny z Guacamole, odstranit nepotřebný kód a upravit vše k funkčnosti v prostředí s použitím CORS. 

Guacamole také poskytuje mnohá nastavení a pro ulehčení byla do modulu přidána administrační stránka, která přímo zapisuje do konfiguračního souboru Guacamole a tím eliminuje potřebu připojení na server. 

\section{Ostranění autentizace uživatelů a vytvoření vlastní autorizace}
Guacamole v základu poskytuje API pro správu uživatelských účtů a jejich připojení. Jsou implementovány dvě metody, jedna spoléhá na uložení dat do konfiguračního souboru, zatímco druhá pracuje s MySQL databází. Obě rozšiřují třídu \emph{SimpleAuthenticationProvider}, ke které přidávají potřebnou funkcionalitu. 

Pro potřeby tohoto projektu však bylo potřeba vždy pouze jediné připojení, pro jednotlivé uživatele se měnilo pouze jejich přihlašovací jméno. Byla vytvořena nová třída \emph{DrupalAuthenticationProvider}, která načítá adresu serveru a port přímo z konfiguračního souboru pro všechna připojení a vrací je Guacamole k dalšímu zpracování. 

Pro nastavení přihlašovacích údajů je potřeba se nejdříve k serveru přihlásit. Pro minimalizaci úkonů potřebných ke každému připojení ke vzdálené ploše a také k zjednodušení celého procesu si musí každý uživatel portálu uložit své přihlašovací údaje do svého profilu, odkud jsou pak použity při přihlášení ke Guacamole Servletu. K tomu je použito dvou služeb - \emph{connect} a {login}. Proces je znázorněn na diagramu \ref{fig:login_process}. 

Komunikace je započata přímo v klientském prohlížeči z JavaScriptu. Po přístupu na stránku /aspi je nejdříve kontaktována pomocí technolie AJAX stránka /aspi/ajax na portálu, která se připojí ke servletu Login. Guacamole Servlet si vytvoří nové \gls{session} a jeho identifikátor odešle zpět PHP kódu. Pokud vše proběhne v pořádku, je identifikátor sezení odeslán pomocí set-cookie hlavičky zpět stránce /aspi, kde si informaci uloží prohlížeč jako cookie pro pozdější komunikaci. V případě jakékoliv chyby je chybové hlášení vypsáno na obrazovce a zalogováno v systému bez jakýchkoliv dalších pokusů o připojení, uživatel je tedy nucen buď stránku obnovit, nebo kontaktovat správce stránek. Pokud však vše proběhne v pořádku, je inicializována smyčka přímých volání služby \emph{tunnel} poskytované Guacamole Servlet kontejnerem za pomocí technologie CORS. Komunikace probíhá za pomocí Guacamole protokolu a ten je poté konvertován na grafický výstup na obrazovce uživatele, který je od této chvíle schopen pomocí myši a klávesnice ovládat vzdálenou plochu a tím i ASPI.

\begin{figure}[]
  \includegraphics[scale=0.85]{img/login-process-crop.pdf}
  \caption{Proces připojení ke vzdálené ploše}
  \label{fig:login_process}
\end{figure}  

\section{Přesun JavaScript kódu a stylů do modulu Drupalu}
Pro ulehčení správy \gls{session}, které by jinak muselo být synchronizováno mezi dvěma stránkami, byl veškerý kód stránky z Guacamole JAVA projektu přesunut do modulu v drupalu. Do složky js/lib byly přesunuty všechny zdrojové kódy z knihovny Guacamole-js a byla vytvořena nová stránka /aspi poskytující stejnou funkcionalitu jako JAVA modul. Spolu s javascriptem bylo nutné přesunout i základní css styly pro zachování formátu vykreslování okna vzdálené plochy. Jak knihovny, tak styly jsou importovány pouze na stránce /aspi pro snížení náročnosti běhu celé platformy, která by jinak byla nucena načítat nepotřebné soubory. 

Drupal v základu vytváří komplexní html výstup včetně loga, základní struktury a podobně, kdežto guacamole vyžaduje pro vykreslení vzdálené plochy velmi jednoduchou strukturu obsahující prakticky jen plátno pro vykreslování výstupu a nastavení v hlavičce. Zatímco údaje do hlavičky lze přidat jednoduše pomocí implementace hook\_api drupalu (viz. sekce \ref{sec:technologies}), změnu výstupu bylo nutné implementovat pomocí kombinace několika technik. Téma vzhledu \emph{Omega} ve verzi 4.x poskytuje funkcionalitu rozdílných rozvržení (layouts) dostupných pro různé části stránky. Tato možnost lze skombinovat s modulem \emph{Context} a jeho podmodulem \emph{Context Omega}, který poskytuje přemostění mezi modulem a tématem vzhledu a tím i možnost změny rozvržení podle definovaných pravidel. V případě tohoto projektu stačilo nadefinovat pravidlo změny pro url /aspi na které se automaticky přepne rozvržení na minimalistické rozvržení nazvané \emph{Guacamole}. Toto nastavení je i exportováno do konfiguračního modulu esf\_feature.

\section{Implementace CORS}
Pro možnost komunikace mezi více doménami bylo potřeba implementovat CORS (viz. sekce \ref{sec:technologies}). Hlavní změna je u zpracování požadavků na straně serveru, k čemuž musela být do projektu přidána a nastavena knihovna k jeho zpracování. Pro JAVA EE byla využita knihovna CORS filter od [d]zhuvinov  [s]oftware\cite{website:cors-filter}, která tuto implementaci poskytuje ve formě knihovny. Na stránkách je i detailně popsáno nastavení a použití, kdy v připadě tohoto serveru bylo nutno omezit zdroj CORS požadavků na jedinou doménu a více nebylo nutno se o bezpečnost obávat.

Druhá část implementace je umožnění cross-domain požadavků v JavaScriptu. Knihovna Guacamole-js musela být lehce přepsána v místech, kde byla použita volání XMLHttpRequest, která jsou v tomto případě volána asynchronně. Použité řešení bohužel nelze aplikovat na \gls{ie}, neboť XDomainRequest nepodporuje asynchronní zpracování a bude nutné toto řešení ještě přepracovat do funkční podoby.

\section{Implementace úpravy nastavení z administrace portálu}
Guacamole typicky poskytuje nastavení pomocí konfiguračního souboru umístěného buď ve složce instalace, nebo v domácí složce uživatele, který program spouští. Jelikož je ke změně těchto konfiguračních souborů nutné přistupovat přímo k serveru, rozhodl jsem se tuto administrační část přepracovat a zpřístupnit přímo z administrace portálu. Ta nabízí dvě základní stránky. Na stránce Nastavení (/config/esf/settings) je možné upravit základní nastavení projektu jako je URL adresa pro připojení ke Guacamole Servlet službám či právě přístup ke konfiguračnímu souboru. Ten se typicky jmenuje guacamole.properties a nachází se ve složce /srv/guacamole/. Je nutné se ujistit, že jak Guacamole Servlet (TomCat) tak ESF Portál (Apache) mají k souboru přístup a mohou jej upravovat.

Na stránce nastavení Připojení ke vzdálené ploše (config/esf/remote) lze za předpokladu, že je vše správně nastaveno, upravovat nastavení Guacamole samotného a url pro připojení k serveru vzdálené plochy - což je v našem případě ASPI. Dále je nutné nastavit port pro připojení ke Guacd démonovi poskytujícímu proxy pro připojení ke vzdálené ploše, uživatelské jméno a heslo je však převzato z nastavení každého jednotlivého uživatele.

\section{Instalace a konfigurace řešení}

\chapter{Organizace vývojového procesu}
\label{chap:vyvoj}

Veškerý vývoj probíhal na lokálně běžícím serveru. Pro minimalizaci nároků na výkon zařízení byl namísto typicky využívaného http serveru Apache nainstalován Lighttpd poskytující dostačující funkcionalitu při mnohem nižších požadavcích na výkon. Pro databázi byl namísto MySQL použit MySQL Lite, který nepotřebuje k běhu instalaci, ale celá implementace běží pouze nad jedním souborem uloženém na disku – v případě Drupalu mezi daty dané stránky. 

Pro dostupnost komplexního testování i bez využití internetu byl lokálně nainstalována služba Guacamole Proxy (Guacd) a místo připojení ke vzdálenému zařízeni byl virtuálně spuštěn systém Windows Server 2008 SP2, na kterém byla povolena vzdálená plocha pomocí protokolu RDP. Pro JAVA aplikaci byl použit server Apache Tomcat, stejně jako na produkčním serveru.

\begin{table}
  \caption{Porovnání technologií použitých na lokálním a produkčním prostředí}
  \begin{tabular}{ | p{3cm} | p{4cm} | p{4cm} | }
    \hline  
    & Vývojové prostředí & Produkční prostředí \\ \hline
    HTTP Server & Lighttpd & Apache \\ \hline
    SQL databáze & SQLite [OPRAVIT???] & MySQL \\ \hline
    JAVA Servlet Container & Apache Tomcat 7 & Apache Tomcat 7 \\ \hline
    Vzdálená plocha & Windows Server 2008 SP2 (Oracle VM VirtualBox) & ASPI [PŘIDAT VÍCE DETAILŮ] \\ \hline
  \end{tabular}
\end{table}

Protože Drupal je postaven z velké části na konfiguraci uložené v databázi, bylo potřeba vymyslet způsob, jak změny prováděné na lokálním stroji efektivně přenášet do produkčního prostředí. Pro uložení nastavení do konfiguračních souborů byl použit projekt Features (sekce \ref{subsec:features}), který exportuje pomocí funkcionality postkytnuté modulem Ctools nastavení drupalu jako jsou typy obsahu a také pomocí modulu Strongarm nastavení systéu samotného. Všechna tato nastavení jsou pak uložena do modulu Drupalu a mohou pak být jednoduše přenesena na jiné prostředí a také aktualizována v průběhu času a s přibývajícím vývojem. Pro udržení přehlednosti byla konfigurace rozdělena na tři části:
\begin{description}
  \item[ESF Feature (esf\_feature)] základní nastavení systému včetně typů obsahů, vztahů mezi nimi, metody zadávání obsahu (WYSIWYG) a základní prvky zobrazené na stránkách

  \item[ESF Feature UI (esf\_feature\_ui)] administrační rozhraní pro správu obsahu a nádstavba nad modulem Workbench (viz. \ref{subsec:workbench}), poskytující náhled na jednotlivé typy obsahu stránek
  \item[ESF Permissions (esf\_permissions)] definice uživatelských rolí a jejich práv
\end{description}
Moduly pak byly uloženy v repozitáři a tím zajištěno jejich správné verzování. V případě potřeby je možné je jednoduše doručit do produkčního prostředí a nastavení je pak možno načíst z daných modulů, kdy v případě bezproblémové aktualizace (v komponentech popsaných v daném modulu nebyly manuálně provedeny žádné změny) se změny provedou automaticky – jinak musí být přes rozhraní či drush manuálně určeno, která změna se má využít a zda se případně nemá aktualizovat daný modul.

Instalační skript a profil jsou generovány automaticky pomocí modulu Profiler\_Builder. Jeho výstupem je instalační soubor .make a instalační profil. V instalačním souboru .make je vypsán seznam modulů a jejich verzí, určený pro příkaz drush make, který stáhne potřebné projekty z repozitáře Drupalu. Po vytvoření je nutné ručně odstranit moduly esf\_*, které Drush neumí automaticky stáhnout a instalace by selhala. Díky rozdělení na dva hlavní a doplňkový instalační skript je možné definovat url adresy ke knihovnám, které nelze automaticky doplnit a ty zároveň nejsou přepsány opětovným vygenerováním, neboť hlavní instalační soubor je přesunut do kořenové složky a přejmenován na esf.make, který již není nutné aktualizovat. V instalačním profilu esf\_profile jsou obsažena základní nastavení portálu a seznam modulů, které je nutné povolit ke správné funkčnosti stránek. Ke každému modulu lze přiřadit i opravné balíčky (patch), které jsou buď automaticky dohledány na na stránkách Drupalu a jako odkazy přidány do profilu, nebo mohou být přidány dodatečně ručně. Při instalaci jsou automaticky aplikovány na kód. Díky propojení s features se na stránky automaticky dostane i rozšířené nastavení a struktura.

\section{Organizace řízení projektu}
S ohledem na nízký počet zainteresovaných osob a rozsah projektu jsme neimplementovali žádné pokročilé metody projektového řízení a rozhodli jsme se pro jednoduchý seznam úkolů. Jelikož jsme pro uložení zdrojových kódů projektu využili GitHub, bylo nejjednodušší jej rovněž využít pro správu úkolů. Ačkoliv se nemůže rovnat s platformami, specializujícími se jen daný úkol, poskytuje několik základních prvků - problém (issue), milník (milestone) a značku (tag). Značky lze jednoduše využít pro rozlišení mezi úkolem a chybou a také důležitostí problému. Milníky byly využity pro jednoduché plánování a sledování pokroku.

\subsection{Značky}
Typy úkolů
\begin{itemize}
  \item Úkol \emph{(task)} - úkol, který bylo třeba vykonat na projektu
  \item Chyba \emph{(bug)} - chyba nalezená na projektu, kterou bylo potřeba opravit
\end{itemize}
Priorita
\begin{itemize}
	\item Nízká \emph{(0-low)}
	\item Střední \emph{(1-medium)}
	\item Vysoká  \emph{(2-high)}
	\item Kritická \emph{(3-critical)}
\end{itemize}

\subsection{Milníky}
\begin{itemize}
  \item 0.1 | Inicializace - úvodní výzkum týkající se připojení ke vzdálené ploše a dostupných technologií
  \item 0.2 | Drupal modul - vytvoření modulu pro drupal a jeho základní funkcionalita
  \item 0.3 | Struktura a práva - struktura stránek, jejich obsahu a práva uživatelů k jejich použití
  \item 0.4 | Guacamole Drupal - přesun funkcionality z Guacamole do Drupal modulu a jeho propojení s JAVA server aplikací
  \item 0.5 | Test v praxi - změny potřebné k umístění řešení na produkční servery a vytvoření skriptů k automatizaci tohoto procesu
  \item 1.0 | Základní verze - spuštění základní funkční verze
  \item 1.1 | Produkční verze - vyřešení všech problémů, komunikace se stranou klienta a přípravy na reálné spuštění v produkčním prostředí
  \item 1.2 | Údržba - první z verzí, ve kterých se bude dodávat údržba řešení
\end{itemize}


%% Lists of tables and figures, glossary, etc.
\printindex
\printglossary
\listoffigures
\listoftables

%% Bibliography from references.bib
\begingroup
\def\tmpchapter{0}
\renewcommand{\chaptername}{}
\renewcommand{\thechapter}{}
%\addtocontents{toc}{\setcounter{tocdepth}{-1}}
\chapter{Reference}
\renewcommand{\chapter}[2]{}% for other classes

\bibliographystyle{plain}
\bibliography{mvantuch}

%\addtocontents{toc}{\setcounter{tocdepth}{2}}
\endgroup

%% Additional materials
\appendix

%% End of the whole document
\end{document}
